\documentclass[answers]{exam}

% Margins
\usepackage{geometry}
\geometry{left=10mm,right=10mm,%
top=20mm,bottom=20mm}

% More Math Symbols
\usepackage{amssymb}

% Margin notes
\usepackage{marginnote}

% Dashed rules
\usepackage{dashrule}

% Trademark Symbols
\usepackage{textcomp}

% Text in math
\usepackage{amsmath}

% For Diagrams
\usepackage{graphicx}
\usepackage{wrapfig}

% Footnote URLs
\usepackage{hyperref}

% Underlines
\usepackage{soul}

% Tikz
\usepackage{tikz}
\usetikzlibrary{fit, 
                chains,
                positioning,
                shapes.geometric}

% Font
\usepackage{anyfontsize}
%\usepackage[default,regular,black]{sourceserifpro}
\usepackage{tgpagella,eulervm}
\usepackage[T1]{fontenc}

% Mutli-column text
\usepackage{multicol}

% For tables
\usepackage[table]{xcolor}
\usepackage{array}
\usepackage{tablefootnote}

% Short negative symbol in math mode (sub-zero mnemonic)
\newcommand{\sz}{\text{-}}

% Clearer math
\everymath{\displaystyle}

% Degree
\usepackage{gensymb}

% Slanted Parallel
\newcommand{\slparallel}{\mathbin{\!/\mkern-5mu/\!}}

% I like combinatorics to be in the format nCr
\usepackage{leftindex}
\newcommand{\comb}[2]{\leftindex^{#1}{\text{C}}_{#2}}
\newcommand{\perm}[2]{\leftindex^{#1}{\text{P}}_{#2}}

% Probability Trees
\usepackage{forest}

% Setting the fontsize
\newcommand{\normalfontsize}{\fontsize{14pt}{16pt}\selectfont}
\newcommand{\titlefontsize}{\fontsize{16pt}{20pt}\selectfont}

% Solution box formatting
\unframedsolutions
\AtBeginEnvironment{solution}{\color{blue}}
\renewcommand{\solutiontitle}{}
\usepackage{framed}
\definecolor{shadecolor}{RGB}{230, 230, 230}

% Document-specific commands
%% Thanks https://tex.stackexchange.com/a/691776
\newcommand{\bijections}[1]{\begin{tikzpicture}[
    node distance = 1mm and 18mm,
 every fit/.style = {ellipse, draw,
                     minimum width=2cm, inner sep=0pt,
                     node contents={}
                     },
      start chain = going below,
                         ]
 %Draw labels
 \foreach \i/\j [count=\n] in {1/1, 2/2, 3/3}
 {
     \node[on chain]     (a\n) {$\i$};
     \node[right=of a\n] (b\n) {$\j$};
 }

 %Draw arrows
 \foreach \i/\j in {#1}
     \draw[-stealth, semithick] (\i) -- (\j);
 \end{tikzpicture}}
%% Probability Tree: Thanks https://tex.stackexchange.com/a/425271
\forestset{
  ptree/.style={
    for tree={
      grow'=0,
      parent anchor=children,
      child anchor=parent,
      l=3cm,
      s sep = 2em
    },
    before typesetting nodes={
      for tree={
        split option={content}{:}{content, my edge label},
      },
    },
  },
  my edge label/.style={
    if={
      > O_= {n'}{1}
    }{
      edge label={node [midway, below] {#1} }
    }{
      edge label={node [midway, above] {#1} }
    },
  }
}

\begin{document}

\noindent
\titlefontsize Discrete Structures for Computing -- SSP -- Final Exam 2024\\
\normalfontsize Dr. Amin Shoukry, Dr. Yasmine Abouelseoud\\Recreated and solved by a student
\hrule
\begin{questions}
    \question Prove by induction $\sum \limits_{k = 1}^{n} (k \cdot k!) = (n + 1)! - 1$\\ Note: $n \ge 1$
    \begin{shaded}
        \begin{solutionorbox}
            Let $P(n): \sum \limits_{k = 1}^{n} (k \cdot k!) = (n + 1)! - 1$\\[+.7em]
            Prove $P(n) ~ \forall n \ge 1 $\\
            1. \textbf{Base Case}, let $n = 1$ and prove that $P(1)$ is true.
            \begin{align*}
                \sum \limits_{k = 1}^{1} (k \cdot k!) & = (n + 1)! - 1 \tag{Base Case} \\
                1 \times 1!                           & = (1+1)! - 1 \tag{Substitute}  \\
                1                                     & = 2 - 1 = 1 \tag{Proven}
            \end{align*}
            2. \textbf{Inductive Step}, prove that $P(x) \rightarrow P(x+1) \quad \forall x \ge 1$\\
            3. \textbf{Inductive Hypothesis}, $n = x$\\
            Assume $P(x)$, prove $P(x+1)$
            \begin{align*}
                \sum \limits_{k = 1}^{x} (k \cdot k!)            & = (x + 1)! - 1 \tag{Assume $P(x)$}                                               \\
                P(x+1) = \sum \limits_{k = 1}^{x+1} (k \cdot k!) & = ((x + 1) + 1)! -1 \tag{R.T.P $P(x+1)$}                                         \\
                \sum \limits_{k = 1}^{x+1} (k \cdot k!)          & = (x + 2)! -1 \tag{Simplify}                                                     \\
                \text{L.H.S}                                     & = \sum \limits_{k = 1}^{x} (k \cdot k!) + (x+1) \cdot (x+1)! \tag{Expand Series} \\
                                                                 & = (x + 1)! - 1 + (x+1) \cdot (x+1)! \tag{Sub. from $P(x)$}                       \\
                                                                 & = (x + 1)! \cdot (1 + (x+1)) - 1 \tag{Simplify}                                  \\
                                                                 & = (x + 1)! \cdot (x + 2) - 1 \tag{Simplify}                                      \\
                                                                 & = (x + 2)! - 1 \tag{Factorial Simplification}                                    \\
                                                                 & = \text{R.H.S} \tag{Proven}
            \end{align*}
        \end{solutionorbox}
    \end{shaded}
    \question Is the function $f : \mathbb{Z} \longrightarrow \mathbb{Z}, f(x) = 4 - 3x$, onto? prove your answer.
    \begin{shaded}
        \begin{solutionorbox}
            \textbf{NO}\\
            (Counter-example)\\
            Find $n \in \mathbb{Z}$ which has no pre-image under $f$ (in range, but not in co-domain).\\
            Let $n = 0$:
            \begin{align}
                0    & = 4 - 3x \notag      \\
                -4   & = -3x \notag         \\
                x    & = \frac{4}{3} \notag
            \end{align}
            $x = \frac{4}{3} \notin \mathbb{Z}$, hence $f$ is NOT onto.
        \end{solutionorbox}
    \end{shaded}
    \question Prove that the function $f : \mathbb{R} - \{ 2 \} \longrightarrow \mathbb{R} - \{ 5 \}, f(x) = \dfrac{5x + 1}{x - 2}$, is one-to-one.
    \begin{shaded}
        \begin{solutionorbox}
            Definition of a one-to-one function: $f(n) = f(m)$ if, and only if, $n = m$.\\
            (Direct Proof)\\
            $n = m \rightarrow f(n) = f(m)$ does not need proving.\\
            R.T.P: $f(n) = f(m) \rightarrow n = m$
            \begin{align*}
                f(n)                                      & = f(m) \notag                                      \\
                \frac{5n+1}{n-2}                          & = \frac{5m+1}{m-2} \tag{Substitute}                \\
                (m-2) (5n + 1)                            & = (n-2) (5m+1) \notag                              \\
                \underline{5mn} + m - 10n - \underline{2} & = \underline{5mn} + n - 10m - \underline{2} \notag \\
                m - 10n                                   & = n - 10m \notag                                   \\
                11m                                       & = 11n \notag                                       \\
                m                                         & = n \tag{Proven}
            \end{align*}
        \end{solutionorbox}
    \end{shaded}
    \question Draw diagrams of all bijections $f: X \longrightarrow X, X = \{ 1, 2, 3 \}$
    \begin{shaded}
        \begin{solutionorbox}
            \bijections{a1/b1, a2/b2, a3/b3}
            \bijections{a1/b1, a2/b3, a3/b2}
            \bijections{a1/b2, a2/b1, a3/b3}
            \bijections{a1/b2, a2/b3, a3/b1}
            \bijections{a1/b3, a2/b1, a3/b2}
            \bijections{a1/b3, a2/b2, a3/b1}
        \end{solutionorbox}
    \end{shaded}
    \question How many ways to arrange the word \textbf{TENNESSEE}?
    \begin{shaded}
        \begin{solutionorbox}
            (9 Letters: 1 T, 2 N, 2 S, 4 E) Number of ways: $\comb{9}{4} \times \comb{5}{2} \times \comb{3}{2} \times \comb{1}{1}$
        \end{solutionorbox}
    \end{shaded}
    \question How many ways to distribute 8 identical balls among 4 people?
    \begin{shaded}
        \begin{solutionorbox}
            $\binom{8 + 4 - 1}{8} = \comb{11}{8}$
        \end{solutionorbox}
    \end{shaded}
    \question There are 9 players, among them are 3 good front row players. How many teams of five players can you form if at least one good front row player must be included?
    \begin{shaded}
        \begin{solutionorbox}
            Let $F = $ good front row players, $N = $ normal (other) players.\\[+1em]
            \begin{forest}
                for tree={grow=0,l=1cm,child anchor = west,parent anchor= west,anchor = west,calign=fixed edge angles, s sep+=1em}
                [\phantom{}
                [{$3F,2N = \comb{3}{3} \times \comb{6}{2}$}]
                [{$2F,3N = \comb{3}{2} \times \comb{6}{3}$}]
                [{$1F,4N = \comb{3}{1} \times \comb{6}{4}$}]
                ]
            \end{forest}
            $\qquad \text{Total} = \comb{3}{3} \times \comb{6}{2} + \comb{3}{2} \times \comb{6}{3} + \comb{3}{1} \times \comb{6}{4} = \comb{9}{5} - \comb{6}{5}$
        \end{solutionorbox}
    \end{shaded}
    \question You are going to use the 7 characters $a, b, c, d, e, f, g$ to make an ordered list of characters.
    \begin{parts}
        \part How many lists can you make if no repetition is allowed?
        \begin{shaded}
            \begin{solutionorbox}
                $7!$
            \end{solutionorbox}
        \end{shaded}
        \part How many lists can you make if repetitions are allowed?
        \begin{shaded}
            \begin{solutionorbox}
                $7^7$
            \end{solutionorbox}
        \end{shaded}
        \part How many lists can you make if at least one character should be repeated?
        \begin{shaded}
            \begin{solutionorbox}
                $7^7 - 7!$
            \end{solutionorbox}
        \end{shaded}
    \end{parts}
    \pagebreak
    \question You are going to build a binary string of length $8$.
    \begin{parts}
        \part How many strings end with a $1$?
        \begin{shaded}
            \begin{solutionorbox}
                $2^7$
            \end{solutionorbox}
        \end{shaded}
        \part How many strings have exactly four ones?
        \begin{shaded}
            \begin{solutionorbox}
                $\comb{8}{4}$
            \end{solutionorbox}
        \end{shaded}
        \part How many strings end with $1$ or have exactly four ones?
        \begin{shaded}
            \begin{solutionorbox}
                $2^7 + \comb{8}{4} - \comb{7}{3}$
            \end{solutionorbox}
        \end{shaded}
    \end{parts}
    \question If 10 contestants are ranked without ties, how many ways are there for first, second, third places?
    \begin{shaded}
        \begin{solutionorbox}
            $\perm{10}{3}$
        \end{solutionorbox}
    \end{shaded}
    \question Find the coefficient of $x^8 \cdot y^5$ in $(4x^2 - 7y)^9$
    \begin{shaded}
        \begin{solutionorbox}
            \textbf{Binomial theorem}
            \begin{align*}
                (4x^2 - 7y)^9 & = \sum_{k = 0}^{9} \comb{9}{k} (4x^2)^k \cdot (\sz7y)^{9-k} \tag{General Term} \\
                2k = 8 \tag{Power of $x$}                                                                      \\
                k = 4 \notag                                                                                   \\
                \comb{9}{4} \cdot (4x^2)^4 \cdot (\sz7y)^{9-4} \tag{Substitute}                                \\
                \comb{9}{4} \cdot 4^4 x^8 \cdot (\sz7^5) \cdot y^5 \tag{Simplify}                              \\
                \comb{9}{4} \cdot 4^4 \cdot (\sz7^5) \cdot x^8 \cdot y^5 \tag{Re-order}                        \\
                \comb{9}{4} \cdot 4^4 \cdot (\sz7^5) \tag{Isolate Coefficient}
            \end{align*}
        \end{solutionorbox}
    \end{shaded}
    \pagebreak
    \question In a certain population, insurance policy holders are either accident prone or non-accident prone. The probability for an accident prone to have an accident within this year is $0.4$. The probability for a non-accident prone to have an accident within this year is $0.2$. If $30\%$ of the population is accident prone:
    \begin{shaded}
        \begin{solutionorbox}
            Let $p =$ is accident prone, $p^c =$ is non-accident prone, $a =$ will have an accident this year, $a^c =$ will not have an accident this year.\\
            \begin{forest}
                ptree
                [\phantom{}
                [{$p$}:0.3
                [{$a$}:0.4]
                [{$a^c$}:0.6]
                ]
                [{$p^c$:0.7}
                    [{$a$}:0.2]
                    [{$a^c$}:0.8]
                ]
                ]
            \end{forest}
        \end{solutionorbox}
    \end{shaded}
    \begin{parts}
        \part What is the probability of a policy holder to get into an accident within this year?
        \begin{shaded}
            \begin{solutionorbox}
                $\text{P}(a) = 0.3 \times 0.4 + 0.7 \times 0.2
                    = 0.26$
            \end{solutionorbox}
        \end{shaded}
        \part What is the probability of a policy holder to be accident prone if he had an accident within this year?
        \begin{shaded}
            \begin{solutionorbox}
                $\text{P}(p | a) = \frac{\text{P}(p \cap a)}{\text{P}(a)} = \frac{0.3 \times 0.4}{0.26} = \frac{6}{13} \approx 0.46$
            \end{solutionorbox}
        \end{shaded}
    \end{parts}
\end{questions}

\end{document}