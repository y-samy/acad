\documentclass[12pt]{exam}
\RequirePackage{amssymb, amsfonts, amsmath, latexsym, verbatim, xspace, setspace, mathpazo, multicol, xcolor}


% Font
\usepackage{anyfontsize}
%\usepackage[default,regular,black]{sourceserifpro}
%\usepackage{tgpagella,eulervm}
\usepackage[T1]{fontenc}


\usepackage{geometry}
\geometry{left=0.38in,right=0.38in,%
bindingoffset=0mm, top=1em,bottom=4em}
\usepackage[none]{hyphenat}

% Here's where you edit the Class, Exam, Date, etc.
\newcommand{\model}{MODEL 2 [RECREATED]}
\newcommand{\examnum}{Midterm Exam SSP EMPx17 Fall 2023}
\newcommand{\questionpreamble}[1]{\uplevel{[Qu.] #1}}
\newcommand{\newquestion}[1]{\fontsize{12}{14}\selectfont\uplevel{#1[Qu.]}\fontsize{14}{16}\selectfont}
\newcommand{\mat}[1]{\mathrm{#1}}
\pointformat{[\thepoints]}
\pointpoints{pt}{pts}
\renewcommand{\choicelabel}{(\alph{choice})}
\begin{document}


\footrule
\firstpagefooter{\examnum}{}{Page \thepage of \numpages}
\runningfooter{\model ~~~~~~~~ \examnum}{}{Page \thepage of \numpages}

\setlength{\columnseprule}{.1pt}
\def\columnseprulecolor{\color{black}}
\setlength{\columnsep}{2cm}

\begin{multicols*}{2}
\noindent
\textbf{\examnum} \\
\textbf{[Calculator is not allowed]} \\
\rule[1ex]{0.45\textwidth}{.1pt}
\fontsize{14}{16}\selectfont
\begin{questions}
\addpoints
\questionpreamble{If $\mat{A = LU}$ is an $\mat{LU}$ factorization of $\mat{A}$ where $\mat{A = } \begin{bmatrix}
    2 & 0 & 2\\ 
    4 & 3 & 5\\
    -6 & 9 & 0
  \end{bmatrix}$, then}
\question[2] The matrix $\mat{U =} ...$\\
\begin{oneparchoices}
    \begin{tabular}{r c r c}
        \choice & $\begin{bmatrix}
            1 & 0 & 1\\
            0 & 3 & 1\\
            0 & 0 & 3
        \end{bmatrix}$
        &
        \choice & $\begin{bmatrix}
            2 & 0 & 2\\
            0 & 3 & 5\\
            0 & 0 & -15
        \end{bmatrix}$\\
        \choice & $\begin{bmatrix}
            2 & 0 & 2\\
            0 & 3 & 1\\
            0 & 0 & 3
        \end{bmatrix}$
        &
        \choice & $\begin{bmatrix}
            1 & 0 & 1\\
            0 & 3 & 1/3\\
            0 & 0 & 1
        \end{bmatrix}$\\
        \choice & $\begin{bmatrix}
            1 & 0 & 1\\
            0 & 3 & 1\\
            0 & 0 & 1
        \end{bmatrix}$
        &
        \choice & $\begin{bmatrix}
            2 & 0 & 2\\
            0 & 3 & 1\\
            0 & 0 & 1
        \end{bmatrix}$
    \end{tabular}
\end{oneparchoices}
\question[1] and $\mat{L =} ...$\\
\begin{oneparchoices}
    \begin{tabular}{r c r c}
        \choice & $\begin{bmatrix}
            1 & 0 & 0\\
            2 & 1 & 0\\
            -3 & 3 & 1
        \end{bmatrix}$
        &
        \choice & $\begin{bmatrix}
            1 & 0 & 0\\
            -2 & 1 & 0\\
            3 & -3 & 1
        \end{bmatrix}$\\
        \choice & $\begin{bmatrix}
            1 & 0 & 0\\
            4 & 1 & 0\\
            -6 & 3 & 1
        \end{bmatrix}$
        &
        \choice & $\begin{bmatrix}
            1 & 0 & 0\\
            -4 & 1 & 0\\
            6 & -3 & 1
        \end{bmatrix}$\\
        \choice & $\begin{bmatrix}
            1 & 0 & 0\\
            4 & 1 & 0\\
            -6 & 9 & 1
        \end{bmatrix}$
        &
        \choice & $\begin{bmatrix}
            1 & 0 & 0\\
            -4 & 1 & 0\\
            6 & -9 & 1
        \end{bmatrix}$\\
    \end{tabular}
\end{oneparchoices}
\ \\[-2.8em]
\uplevel{\vspace*{-.75\baselineskip}\noindent\rule{\linewidth}{0.5pt}}
\ \\[-3.3em]
\newquestion{}
\question[1] If $\mat{A_{n \times n}}$ is symmetric and $\mat{B_{n \times n}}$ is skew symmetric, then $\mat{2B^t + 3A + 2B}$ is \dots\\
\begin{oneparchoices}
    \choice symmetric\\
    \choice skew symmetric\\
    \choice diagonal\\
    \choice upper triangular\\
    \choice lower triangular\\
    \choice general
\end{oneparchoices}
\vfill\mbox{}
\columnbreak
\leftskip=-0.75cm
\uplevel{\leftskip=-0.75cm \noindent\rule{\linewidth}{0.5pt}}
\ \\[-3.3em]
\questionpreamble{\leftskip=-0.75cm Given the system of equations\\[-1em]
\begin{center}
    \begin{tabular}{c}
        $x-y+2z=1$\\$y-z=2$\\$x-y+3z=-1$
    \end{tabular} 
\end{center}}
\question[2] Calculating the inverse of the coefficient matrix using Gauss-Jordan technique in the intermediate step $\begin{bmatrix}
    1 & -1 & 0 & 3 & 0 & a\\
    0 & 1 & 0 & -1 & 1 & 1\\
    0 & 0 & 1 & b & 0 & 1
\end{bmatrix}$, the values of $a$ and $b$ are \dots\\
\begin{oneparchoices}
    \choice $a = -2\ ,\ b =-1$\\
    \choice $a=0\ ,\ b =-1$\\
    \choice $a = 1\ ,\ b=-1$\\
    \choice $a = -2\ ,\ b=0$\\
    \choice $a = 0\ ,\ b=1$\\
    \choice $a = 0\ ,\ b=0$\\
\end{oneparchoices}
\question[2] The solution $(x,y,z)$ is \dots\\
\begin{oneparchoices}
    \begin{tabular}{r l c l}
        \choice & $(5,0,-2)$ & \choice & $(3,0,-2)$\\
        \choice & $(2,0,-2)$ & \choice & $(5,0,-1)$\\
        \choice & $(3,0,0)$ & \choice & $(3,0,-1)$\\
    \end{tabular}
\end{oneparchoices}
\uplevel{\leftskip=-0.75cm \noindent\rule{\linewidth}{0.5pt}}
\ \\[-3.3em]
\newquestion{\leftskip=-0.75cm}
\question[1] Let $\mat{A}$ be an $\mat{n} \times \mat{n}$ matrix.\ \ \ Suppose that the system $\mat{Ax = b}$ is inconsistent for some $\mat{B} \in \mathbb{R}^\mat{n}$, then the system $\mat{Ax} = 0$ \dots\\
\begin{oneparchoices}
    \choice has two solutions\\
    \choice has unique solution\\
    \choice is inconsistent\\
    \choice has infinite solutions\\
    \choice has three solutions\\
    \choice has four solutions\\[-2.4em]
\end{oneparchoices}
\uplevel{\leftskip=-0.75cm \noindent\rule{\linewidth}{0.5pt}}
\ \\[-3.3em]
\newquestion{\leftskip=-0.75cm}
\question[1] Let $\mat{A}$ be a $7 \times 6$ matrix such that $\mat{Ax=b}$ has unique solution for some $\mat{b} \in \mathbb{R}^7$ then the rank of $\mat{A}$ equals\\
\begin{oneparchoices}
    \begin{tabular}{r c r c r c}
        \choice & 7 & \choice & 6 & \choice & 1\\
        \choice & 0 & \choice & 5 & \choice & 3
    \end{tabular}
\end{oneparchoices}
\end{questions}

\newpage
\begin{questions}
    \newquestion{}
    \question[2] Given the system of equations\\$x + 2y -3z = 0\\2x+y+z=0\\x-y+kz=0$\\where $k$ is a scalar value.\\The value of $k$ for which the following system of linear equations has a non-trivial solution is \dots\\
    \begin{oneparchoices}
        \begin{tabular}{r c r c}
            \choice & $\mathbb{R} - {4}$ & \choice & 
        \end{tabular}
    \end{oneparchoices}
\end{questions}

\end{multicols*}


\end{document}
