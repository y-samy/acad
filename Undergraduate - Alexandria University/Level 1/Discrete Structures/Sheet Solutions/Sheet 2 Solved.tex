\documentclass[a4paper,12pt]{article}

% Margins
\usepackage{geometry}
\geometry{left=25mm,right=25mm,%
bindingoffset=0mm, top=20mm,bottom=20mm}

% Therefore and because
\usepackage{amssymb}

% Text in math
\usepackage{amsmath}

% Underlines
\usepackage{soul}

% Font
\usepackage{anyfontsize}
\usepackage[default,regular,black]{sourceserifpro}
\usepackage[T1]{fontenc}

% Mutli-column text
\usepackage{multicol}

% For tables
\usepackage[table]{xcolor}
\usepackage{array}

% Title
\newcommand{\linia}{\rule{\linewidth}{0.5pt}}
\makeatletter
\renewcommand{\maketitle}{
\begin{center}
\vspace{2ex}
{\huge \textbf{\@title}}
\vspace{1ex}
\\
\linia\\
\textsf{\@date \hfill
\@author}
\vspace{4ex}
\end{center}
}
\makeatother

% Tilde logical not symbol
\newcommand*{\ltnot}{\mathord{\sim}}
% Short negative symbol in math mode (sub-zero mnemonic)
\newcommand{\sz}{\text{-}}

\begin{document}
\title{Discrete Structures --- Sheet 2 Solution}
\author{Youssef Samy}
\maketitle
% Setting the fontsize
\fontsize{14}{16}\selectfont

\noindent \textbf{Q.~1} Let $\mathbb{R}$ be the domain of the predicates ``$x > 1$'', ``$x>2$'', ``$|x|>2$'' and ``$x^2 > 4$''. Which of the following are true and which are false?\\

\bgroup
\def\arraystretch{1.5}
{\rowcolors{2}{white!90!gray!70}{white!70!gray!70}
    \begin{tabular}{l|l l}
        \textbf{STATEMENT}              & \textbf{VERDICT} \\
        (a) $x>2 \Rightarrow x > 1$     & True             \\
        (b) $x^2 > 4 \Rightarrow x > 2$ & False            \\
        (c) $x>2 \Rightarrow x^2 > 3$   & True             \\
        (d) $x^2>4 \Rightarrow |x| > 2$ & True
    \end{tabular}}
\egroup
\ \\

\textbf{REASONING}
\begin{itemize}
    \item[(a)] All values of $x$ that make $x>2$ true also make $x>1$ true
    \item[(b)] $\because x^2 > 4$ can be expressed as $(x > 2 \text{ and } x<\sz2)$.\\
          $\because$ All elements in the truth set for $x<\sz2$ don't belong in the truth set for $x>2$\\
          $\therefore$ The truth set for $x^2 >4$ is not a subset of the truth set for $x > 2$
    \item[(c)] $\because x^2 > 4$ can be expressed as $(x > 2 \text{ and } x<\sz2)$.\\
          $\therefore$ All elements in the truth set for $x>2$ also appear in the truth set for $x^2>4$
    \item[(d)] Both truth sets are identical as both expressions can be rewritten as $(x > 2 \text{ and } x<\sz2)$
\end{itemize}

\begin{center}
    \rule{6cm}{0.4pt}
\end{center}

\noindent \textbf{Q.~2} Find counterexamples to show that the following statements are false:
\begin{itemize}
    \item[(a)] $\forall$ positive integers $m$ and $n$, $m \cdot n \geq m + n $\\
        \textsl{\ul{(counterexample)}} $m=1, n=2 \rightarrow 1 \times 2 \ngeq 1 + 2$
    \item[(b)] $\forall$ real numbers $x$ and $y$, $\sqrt{x+y} = \sqrt{x} + \sqrt{y}$\\
        \textsl{\ul{(counterexample)}} $x=1, y=1 \rightarrow \sqrt{1+1} \neq \sqrt{1} + \sqrt{1}$
    \item[(c)] $\forall x \in \mathbb{R}, x > \frac{1}{x}$\\
        \textsl{\ul{(counterexample)}} $x = 0.5 \rightarrow 0.5 \ngtr \frac{1}{0.5}$
    \item[(d)] $\displaystyle \forall a \in \mathbb{Z}, \frac{a-1}{a}$ is not an integer\\
        \textsl{\ul{(counterexample)}} $a = 1 \rightarrow \frac{0}{1} = 0, 0 \in \mathbb{Z}$
\end{itemize}

\begin{center}
    \rule{6cm}{0.4pt}
\end{center}
\newpage

\noindent \textbf{Q.~3} Let $\text{\textsl{\textbf{D}}} = \{\sz48, \sz14, \sz8, 0, 1, 3, 16, 23, 26, 32, 36\}$ Determine which of the following statements are true and which are false. Provide counterexamples for those statements that are false.

\noindent
\bgroup
\def\arraystretch{1.5}
{\rowcolors{2}{white!90!gray!70}{white!70!gray!70}
    \begin{tabular}{m{23em} | l }
        \textbf{STATEMENT}                                                                                           & \textbf{VERDICT} \\
        (a) $\forall x \in \text{\textsl{\textbf{D}}}$, if $x$ is odd, then $x>0$                                    & True             \\
        (b) $\forall x \in \text{\textsl{\textbf{D}}}$, if $x$ is less than 0, then $x$ is even                      & True             \\
        (c) $\forall x \in \text{\textsl{\textbf{D}}}$, if $x$ is even then $x \leq 0$                               & False : $x = 16 \in \text{\textsl{\textbf{D}}}$ \\
        (d) $\forall x \in \text{\textsl{\textbf{D}}}$, if the ones digit of $x$ is 2, then the tens digit is 3 or 4 & True \\
        (e) $\forall x \in \text{\textsl{\textbf{D}}}$, if the ones digit of $x$ is 6, then the tens digit is 1 or 2 & False : $x = 36 \in \text{\textsl{\textbf{D}}}$ \\
    \end{tabular}}
\egroup
\begin{center}
    \rule{6cm}{0.4pt}
\end{center}

\noindent \textbf{Q.~4} Write negations for each of the following statements:
\begin{itemize}
    \item[(a)] $\forall$ real numbers $x$, if $x > 3$ then $x^2 > 9$\\
    (symbolic) $\forall x \in \mathbb{R} (x>3 \implies x^2 > 9)$ \\
    \textsl{\ul{(negation)}} $\exists x \in \mathbb{R}$ such that $x>3$ and $x^2 \leq 9$ \\
    \textsl{\ul{(symbolic)}} $\exists x \in \mathbb{R} (x>3 \land \ltnot (x^2 > 9))$

    \item[(b)] $\forall x \in \mathbb{R}$ if $x(x+1) > 0$ then $x>0$ or $x<\sz1$\\
    (symbolic) $\forall x \in \mathbb{R} ( x(x+1) > 0 \implies (x>0 \lor x< \sz 1))$\\ 
    \textsl{\ul{(negation)}} $\exists x \in \mathbb{R}$ such that $x(x+1)>0$ and $x \leq 0$ and $x \ge \sz1$\\
    \textsl{\ul{(symbolic)}} $\exists x \in \mathbb{R} (x(x+1) \land ())$
    \item[(c)] $\forall$ integers $a, b,$ and $c$, if $a-b$ is even and $b-c$ is even, then $a-c$ is even\\
    \textsl{\ul{(negation)}} $\exists $
\end{itemize}

\end{document}