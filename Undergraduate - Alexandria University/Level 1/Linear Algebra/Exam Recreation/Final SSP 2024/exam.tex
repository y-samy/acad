\documentclass[12pt, answers]{exam}
\RequirePackage{amssymb, amsfonts, amsmath, latexsym, verbatim, xspace, setspace, mathpazo, multicol, xcolor}


% Multicols
\usepackage{multicol}

% Color
\usepackage{xcolor}


% Font
\usepackage{anyfontsize}
%\usepackage[default,regular,black]{sourceserifpro}
%\usepackage{tgpagella,eulervm}
\usepackage[T1]{fontenc}

\usepackage{tabto}

\usepackage{ifthen}
\usepackage{mathtools}

\makeatletter
\newenvironment{noparchoices}%
  {%
    \setcounter{choice}{0}%
    \def\choice{%
      \refstepcounter{choice}%
      \choicelabel\ 
      % No need to put the following into a token string; we just put
      % the choicelabel onto the page, so we're at the spot whose page
      % number we want to record:
      \questionobject@pluspagecheck
    }% choice
    \def\CorrectChoice{%
      \refstepcounter{choice}%
      \choicelabel\
      % No need to put the following into a token string; we just put
      % the choicelabel onto the page, so we're at the spot whose page
      % number we want to record:
      \questionobject@pluspagecheck
      \ifprintanswers
      \CorrectChoice@Emphasis
        \fi
    }% CorrectChoice
    \let\correctchoice\CorrectChoice
    % If we're continuing the paragraph containing the question,
    % then leave a bit of space before the first choice:
    \ignorespaces
  }%
\makeatother


\usepackage{geometry}
\geometry{left=0.38in,right=0.38in,%
bindingoffset=0mm, top=1em,bottom=4em}
\usepackage[none]{hyphenat}


\newcommand{\ibmatrix}[1]{\begin{bmatrix}#1\end{bmatrix}} % inline bmatrix

\newcommand{\separator}{\noindent{\rule[1ex]{0.45\textwidth}{.1pt}}}
\newcommand{\newquestion}[3][x 1 mark(s)]{
    {\fontsize{12}{14}\selectfont
    \uplevel{\separator \\[-2mm] [Qu.]
    \fontsize{14}{16}\selectfont
    \ifthenelse{\equal{#2}{}}{[1 mark(s)]}{[#2 #1]}
    \ifthenelse{\equal{#3}{}}{\ \\[-7mm]}{\\#3\\[-1em]}}
    }}
\renewcommand{\choicelabel}{(\alph{choice})}

\newcommand{\model}{Version 9}
\newcommand{\examnum}{SSP Linear Algebra Final Exam}
\newcommand{\examdate}{Fall 2024}


\renewcommand{\questionshook}{
    \setlength{\leftmargin}{2em}
    \setlength{\linewidth}{.45\textwidth}
}
\renewcommand{\questionlabel}{{\fontsize{12}{14}\selectfont\thequestion.}}
\renewcommand{\choicelabel}{(\alph{choice})}

\newcommand{\nextcol}{\vfill\mbox{}\columnbreak}

\usepackage{listofitems}

{
    \setsepchar{\\/,}
    \greadlist\mcq{
        0,0,0,0,0,1 \\%1
        0,0,0,0,1,0 \\%2
        1,0,0,0,0,0 \\%3
        0,0,0,0,1,0 \\%4
        0,1,0,0     \\%5
        0,1,0,0,0,0 \\%6
        0,1,0,0     \\%7
        1,0         \\%8
        1,0         \\%9
        1,0,0,0,0   \\%10
        0,0,1,0     \\%11
        0,1,0,0,0   \\%12
        0,0,0,1,0   \\%13 
        1,0,0,0     \\%14
        0,0,0,1     \\%15
        1,0,0,0,0,0 \\%16
        1,0,0,0,    \\%17
        0,0,0,0,0,1 \\%18
        0,0,0,0,0,1 \\%19
        1,0,0,0     \\%20
        0,0,1,0,0,0 \\%21
        0,1,0,0,0,0 \\%22
        0,1,0,0,0,0 \\%23
        0,0,0,0,0,1 \\%24
        0,1,0,0     \\%25
        0,1         \\%26
        1,0         \\%27
        0,0,1,0     \\%28
    }
}

\newcommand{\printscheme}{
    {
        \fontsize{10}{10}\selectfont
        \hspace*{-\leftmargin}\makebox[0pt][l]{
            Answer Points:
            \foreachitem\currchoice\in\mcq[\thequestion]{(\@alph{\currchoicecnt}) = \currchoice\ifthenelse{\equal{\currchoicecnt}{\listlen\mcq[\thequestion]}}{}{, }}
        }\\[-8mm]
    }
}

\usepackage{etoolbox}
\ifprintanswers
\AfterEndEnvironment{oneparchoices}{\\\printscheme}
\AfterEndEnvironment{noparchoices}{\\\printscheme}
\fi


\begin{document}


\footrule
\firstpagefooter{\model ~~~~~~~~ \examnum~- \examdate}{}{Page \thepage~of \numpages}
\runningfooter{\model ~~~~~~~~ \examnum~- \examdate}{}{Page \thepage~of \numpages}

\setlength{\columnseprule}{.1pt}
\def\columnseprulecolor{\color{black}}

\pointformat{}
\pointsinmargin

\begin{multicols*}{2}
	\noindent
	\textbf{Final SSP Exam - Linear Algebra - \examdate} \\
	RE-CREATED\\
	\textbf{No. of Questions : \numquestions}\\[-8mm]
	\begin{questions}
		\fontsize{14}{16}\selectfont
		\addpoints
		\begingradingrange{exam}
		\newquestion{}{}
        
		\question[1] If matrix $\mathbf{A}$ and $\mathbf{B}$ are similar matrices, and $\mathbf{A}$ is diagonalizable, then so is matrix $\mathbf{B}$

		\begin{oneparchoices}
			\choice True \quad \choice False
		\end{oneparchoices}

		\newquestion{}{}

		\question[1] If $H$ is the set of all polynomials of the form\\
		$(a-2b) + (2a-4b)t + (2b-a)t^2$ where $a,b \in \mathbb{R}$,\\
		then $\mathrm{dim}H$ is equal to \dots

		\begin{oneparchoices}
			\choice $2$ \quad \choice $3$\\
			\choice $1$ \quad \choice not a subspace
		\end{oneparchoices}

		\newquestion{}{}

		\question[1] Consider the following\\
		$T : \mathbb{R}^2 \rightarrow \mathbb{R}^2$ such that\\
		$T(2,2) = (8,-5), T(4,4) = (3,-2)$.\\
		Then

		\ifprintanswers
		\def\tempsize{+3mm}
		\else
		\def\tempsize{+1em}
		\fi
		\begin{oneparchoices}
			\choice $T$ is a linear transformation and one to one.	\\[\tempsize]
			\choice $T$ is a linear transformation and isomorphic.	\\[\tempsize]
			\choice $T$ is not a linear transformation.				\\[\tempsize]
			\choice $T$ is a linear transformation and onto.		
		\end{oneparchoices}

		\newquestion{}{}

		\question[1] Let $T : U \rightarrow V$ be a linear transformation. If $range(T) = span \lbrace (1,1,1),(2,2,2),(1,0,0) \rbrace$, and nullity of $T$ is $1$, then $dim(U) = \dots$

		\begin{noparchoices}
			\choice $0$ ~~ \choice $1$ ~~ \choice $5$ ~~ \choice $4$ ~~ \choice $2$ ~~ \choice $3$
		\end{noparchoices}
		

		\nextcol

		\newquestion{4}{
			Consider the following matrix $A =\\ \ibmatrix{
					2 & 2 & 1\\
					1 & 3 & 1\\
					1 & 2 & 2
				}$
		}

		\question[1] Which of the following is an eigenvalue of $A$

		\begin{noparchoices}
			\begin{tabular}{cccccc}
				\choice & $0$  & \choice & $-3$ & \choice & $2$ \\
				\choice & $-1$ & \choice & $-2$ & \choice & $1$
			\end{tabular}
		\end{noparchoices}

		\question[1] Which of the following is \linebreak an eigenvector of the largest \linebreak eigenvalue of $A$

		\begin{noparchoices}
			{
				\def\arraystretch{1.1}
				\begin{tabular}{c l c l c l}
					\choice & $\ibmatrix{-3 \\-1\\2}$ & \choice &  $\ibmatrix{-2\\-1\\-2}$ & \choice & $\ibmatrix{-1\\0\\-1}$\\
					\choice & $\ibmatrix{0  \\-1\\1}$ & \choice & $\ibmatrix{-2\\-1\\-1}$ &  \choice & $\ibmatrix{1\\1\\1}$
				\end{tabular}
			}
		\end{noparchoices}

		\question[1] Which of the following is an eigenvector of the smallest eigenvalue of $A$

		\begin{noparchoices}
			{
				\def\arraystretch{1.1}
				\begin{tabular}{c l c l c l}
					\choice & $\ibmatrix{-3 \\0\\0}$ & \choice & $\ibmatrix{-3\\1\\1}$ & \choice & $\ibmatrix{-3\\1\\2}$ \\
					\choice & $\ibmatrix{-2 \\1\\1}$ & \choice & $\ibmatrix{2\\1\\-1}$ & \choice & $\ibmatrix{0\\1\\1}$
				\end{tabular}
			}
		\end{noparchoices}

		\question[1] Using the Cayley-Hamilton \linebreak theorem, $A^{-1} = $

		\ifprintanswers
		\def\tempsize{+3mm}
		\else
		\def\tempsize{+1em}
		\fi
		
		\begin{oneparchoices}
			\choice $\frac{1}{5}A^2+\frac{11}{5}I$               \\[\tempsize]
			\choice $\frac{1}{5}A^2-\frac{7}{5}A-\frac{11}{5}I$  \\[\tempsize]
			\choice $\frac{1}{5}A^2-\frac{7}{5}A$                \\[\tempsize]
			\choice $\frac{1}{5}A^2-\frac{7}{5}A+\frac{11}{5}I$  \\[\tempsize]
			\choice $\frac{1}{5}A^2+\frac{7}{5}A+\frac{11}{5}I$  \\[\tempsize]
			\choice $\frac{1}{5}A^2+\frac{7}{5}A-\frac{11}{5}I$
		\end{oneparchoices}

        \newpage

        \newquestion{}{}

        \question[1] If an $n \times n$ matrix has one eigenvalue with multiplicity $n$ (repeated $n$ times), then it is \textbf{NOT} diagonalizable.

		\begin{oneparchoices}
			\choice True \quad \choice False
		\end{oneparchoices}

		\newquestion{}{}
		
		\question[1] Let $T : U \rightarrow V$ be a linear transformation. Which of the following statements is \textbf{\underline{not possible}}?

		\begin{oneparchoices}
			\choice $dim(range(T)) < dim(V).$ \\[+3mm]
			\choice $dim(range(T)) > dim(V).$ \\[+3mm]
			\choice $dim(range(T)) = dim(V).$
		\end{oneparchoices}

		\newquestion{4}{Given the matrix $\mathbf{B} = \ibmatrix{2&1&2\\0&1&2\\0&0&1}$}

		\question[1] How many \textbf{\underline{different}} eigenvalues does matrix $\mathbf{B}$ have?

		\begin{noparchoices}
			\choice 0 ~~\choice 3 ~~\choice 1 ~~ \choice 2
		\end{noparchoices}
		
		\question[1] The basis of the eigenspace of the \textbf{\underline{largest}} eigenvalue of $\mathbf{B}$ is $\lbrace....\rbrace$

		\begin{noparchoices}
			{
				\def\arraystretch{1.1}
				\begin{tabular}{c l c l c l}
					\choice & $\ibmatrix{1\\1\\1}$ & \choice & $\ibmatrix{1\\-1\\0}$ \\\\
					\choice & $\ibmatrix{1\\-1\\0},\ibmatrix{1\\1\\0}$ & \choice & $\ibmatrix{1\\-1\\0},\ibmatrix{1\\0\\0}$ \\\\
					\choice & $\ibmatrix{1\\0\\0}$ & \choice & $\ibmatrix{1\\1\\0}$
				\end{tabular}
			}
		\end{noparchoices}

		\nextcol

		\question[1] The basis of the eigenspace of the \textbf{\underline{largest}} eigenvalue of $\mathbf{B}$ is $\lbrace....\rbrace$

		\begin{noparchoices}
			{
				\def\arraystretch{1.1}
				\begin{tabular}{c l c l c l}
					\choice & $\ibmatrix{1\\-1\\0},\ibmatrix{1\\1\\0}$ & \choice & $\ibmatrix{1\\1\\0}$ \\\\
					\choice & $\ibmatrix{1\\-1\\0}$ & \choice & $\ibmatrix{1\\-1\\0}$ \\\\
					\choice & $\ibmatrix{1\\1\\1}$ & \choice & $\ibmatrix{1\\0\\0},\ibmatrix{1\\0\\0}$
				\end{tabular}
			}
		\end{noparchoices}

		\question[1] Is matrix $\mathbf{B}$ diagonalizable?

		\begin{oneparchoices}
			\choice Yes \quad \choice No
		\end{oneparchoices}

		\newquestion{}{}

		\question[1] Given the set of vectors $S = \linebreak \lbrace 1, t + 5, t^3 + t^2 + 4, t^3 + t^2 + t \rbrace. \linebreak$ Dimension of the space spanned by the set of vectors $S$ is \dots

		\begin{noparchoices}
			\choice 1~~\choice4~~\choice0~~\choice2~~\choice3~~\choice5
		\end{noparchoices}

		\newquestion{2}{Consider the following matrix\\$A = \ibmatrix{1&4\\2&3}$.}

		\question[1] Then, the corresponding\\characteristic equation is\\
		\begin{noparchoices}
			\begin{tabular}{ l r}
				\choice & $\lambda^2 + 4 \lambda + 5 = 0$\\
				\choice & $\lambda^2 + 4 \lambda - 5 = 0$\\
				\choice & $\lambda^2 - 4 \lambda - 8 = 0$\\
				\choice & $\lambda^2 + 4 \lambda + 8 = 0$\\
				\choice & $\lambda^2 - 4 \lambda - 5 = 0$\\
				\choice & $\lambda^2 + 4 \lambda - 8 = 0$
			\end{tabular}
		\end{noparchoices}

		\question[1] For this matrix, using Cayley-Hamilton theorem $A^4$ is equal to\\
		\begin{noparchoices}
			\begin{tabular}{l l l}
				\choice $64A-95I$ & \choice $-64A+95I$ \\
				\choice $104A + 105I$ & \choice $-151A+120I$ \\
				\choice $151A-120I$ & \choice $256A+425I$ \\
			\end{tabular}
		\end{noparchoices}
		\newpage

		\newquestion{}{}

		\question[1] Let $T : \mathbb{R}^3 \rightarrow \mathbb{R}^3$ be a linear transformation defined by $T(x,y,z) = (x,x+y,z)$, then \dots
		
		\begin{oneparchoices}
			\choice $T$ is 1-to-1 but not onto.\\\\
			\choice $T$ is neither 1-to-1 nor onto.\\\\
			\choice $T$ is onto but not 1-to-1.\\\\
			\choice $T$ is isomorphic.\\\\
			\choice $T$ is not a linear transformation.
		\end{oneparchoices}

		\newquestion{}{}
		
		\question[1] if $\ibmatrix{4/3\\1}$ is an eigenvector of $\ibmatrix{2&4\\3&1}$, then the associated eigenvalue is\\
		\begin{noparchoices}
			\begin{tabular}{c l c l c l}
				\choice & 7 & \choice & 2 & \choice & $4/3$\\
				\choice & 1 & \choice & 5 & \choice & $-2$\\
			\end{tabular}
		\end{noparchoices}

		\newquestion{}{}
		\question[1] The set $\begin{Bmatrix}
			\ibmatrix{2\\1\\0\\6}, \ibmatrix{-3\\7\\6\\-18}, \ibmatrix{1\\-5\\-2\\12}
		\end{Bmatrix}$ \linebreak is not a basis for $\mathbb{R}^4$\\
		\begin{oneparchoices}
			\choice True \quad \choice False
		\end{oneparchoices}

		\newquestion{}{}
		\question[1] Let $T : \mathbb{R}^2 \rightarrow \mathbb{R}^2$ be a \linebreak linear transformation where \linebreak $T(x,y) = (x,0)$. The null space \linebreak of $T$ is \dots

		\begin{oneparchoices}
			\choice $(0,1)$\\\\
			\choice $(0,y), y$ is real.\\\\
			\choice $1$\\\\
			\choice $(x,0), x$ is real.
		\end{oneparchoices}

		\nextcol

		\newquestion{4}{Given the following system of linear differential equations:\\$
		x^\prime = 4x + y - 2z\\
		y^\prime = 2x + 2y - z\\
		z^\prime = 2x + y\\
		$ Knowing that the eigenvalues of the coefficient matrix are 1, 2 and 3, answer the following:}
		\question[1] An eigenvector of $\lambda = 1$ is ......
		\begin{oneparchoices}
			\choice $\ibmatrix{1 & 0 & 0}^T$ \quad \choice $\ibmatrix{1 & 2 & 2}^T$ \\
			\choice $\ibmatrix{1 & 1 & 1}^T$ \quad \choice $\ibmatrix{1 & -1 & ?}^T$ \\
			\choice $\ibmatrix{2 & 1 & 1}^T$ \quad \choice $\ibmatrix{1 & -1 & 1}^T$
		\end{oneparchoices}
		\question[1] An eigenvector of $\lambda = 3$ is ......
		\begin{oneparchoices}
			\choice $\ibmatrix{1 & 0 & 0}^T$ \quad \choice $\ibmatrix{1 & 1 & ?}^T$ \\
			\choice $\ibmatrix{1 & -1 & -1}^T$ \quad \choice $\ibmatrix{1 & ? & ?}^T$ \\
			\choice $\ibmatrix{2 & 1 & 1}^T$ \quad \choice $\ibmatrix{1 & 2 & 2}^T$
		\end{oneparchoices}
		\newpage
		\question[1] ?
		\question[1] ?
		\newpage

		\newquestion{2}{If $A = \ibmatrix{-1&2&1\\1&-1&0\\2&3&1}$}

		\question[1] The LU decomposition of $A$ is

		\begin{oneparchoices}
			{
				\def\arraystretch{1.7}
				\choice $\ibmatrix{1&0&0\\-1&1&0\\-2&7&1} \ibmatrix{1&-2&-1\\0&1&1\\0&0&1}$ \\[+1em]
				\choice $\ibmatrix{-1&2&1\\0&1&1\\0&0&-4} \ibmatrix{1&0&0\\-1&1&0\\-2&7&1}$ \\[+1em]
				\choice $\ibmatrix{1&0&0\\1&1&0\\2&-7&1} \ibmatrix{-1&2&1\\0&1&1\\0&0&-4}$ \\[+1em]
				\choice $\ibmatrix{1&0&0\\1&1&0\\2&-7&1} \ibmatrix{1&-2&-1\\0&1&1\\0&0&1}$ \\[+1em]
				\choice $\ibmatrix{1&0&0\\-1&1&0\\-2&7&1} \ibmatrix{-1&2&1\\0&1&1\\0&0&-4}$ \\
			}
		\end{oneparchoices}
		\question[1] The columns of $A$ span $\mathbb{R}^3$

		\begin{oneparchoices}
			\choice True \quad \choice False
		\end{oneparchoices}
		\nextcol

		\newquestion{}{}
		\question[1] For the matrix $A = \ibmatrix{2 & 3\\c&d}$, if the eigenvalues of $A$ are 4 and 8, then $c$ and $d$ are respectively equal

		\begin{noparchoices}
			\begin{tabular}{c l c l c l}
				\choice & 4 and $-10$ & \choice & 5 and 8 \\
				\choice & 3 and $-9$ & \choice & $-3$ and 9 \\
				\choice & 7 and 5 & \choice & $-4$ and 10
			\end{tabular}
		\end{noparchoices}

		\newquestion{}{}
		\question[1] Let $T : \mathbb{R}^5 \rightarrow \mathbb{R}^3$ be the linear transformation defined  by $T(x_1,x_2,x_3,x_4,x_5) =\\ (x_1+x_2, x_2 + x_3 + x_4, x_4 + x_3)$. \linebreak The nullity of the transformation matrix of $T$ is

		\begin{noparchoices}
			\choice 3~~\choice5~~\choice1~~\choice4~~\choice2~~\choice0
		\end{noparchoices}

		\newquestion{}{Consider the augmented matrix of a system of equations given by\\
		$\text{Aug}(A) = \left \lbrack\begin{matrix}
			1 & 2 & 4\\
			0 & 3 & 2\\
			0 & 0 & 1\\
			0 & 0 & -1\\
		\end{matrix} \right \rvert \left . \begin{matrix}
			0 \\ 3 \\ 2 \\1
		\end{matrix} \right \rbrack$
		}

		\question[1] The system has

		\begin{oneparchoices}
			\choice infinite no. of solutions\\
			\choice unique solution\\
			\choice no solution\\
			\choice cannot be determined
		\end{oneparchoices}
		\newpage

		\newquestion{5}{Given the matrices\\
			$\mathbf{M}_1 = \ibmatrix{1 & 1\\2 & 4}, \mathbf{M}_2 = \ibmatrix{0 & 1 \\ 1 & 1},\\[+1mm]
			\mathbf{M}_3 = \ibmatrix{1 & 1\\2 & -1}, \mathbf{M}_4 = \ibmatrix{1 & 2\\3&5}$\\[+1mm]
			If you know that :\\
			{
				\def\arraystretch{1.2}
				$\ibmatrix{
					1 & 0 & 1 & 1\\
					1 & 1 & 1 & 2\\
					2 & 1 & 2 & 3\\
					4 & 1 & -1 & 5
				} \xrightarrow{\text{EF}}
				\ibmatrix{
					1 & 0 & 1 & 1\\
					0 & 1 & 0 & 1\\
					0 & 0 & -5 & 0\\
					0 & 0 & 0 & 0 
				}$
		 	}
		}

		\question[1] The set $\lbrace \mathbf{M}_1,\mathbf{M}_2,\mathbf{M}_3,\mathbf{M}_4 \rbrace$ is

		\begin{oneparchoices}
			\choice dependent \quad \choice independent
		\end{oneparchoices}

		\question[1] Which matrix can be written as a linear combination of $\mathbf{M}_1$ and $\mathbf{M}_2$?

		\begin{oneparchoices}
			\choice Neither $\mathbf{M}_3$ nor $\mathbf{M}_4$ \\\\
			\choice $\mathbf{M}_4$ \\\\
			\choice $\mathbf{M}_3$ \\\\
			\choice Both $\mathbf{M}_3$ and $\mathbf{M}_4$
		\end{oneparchoices}


		\question[1] The set $\lbrace\mathbf{M}_1,\mathbf{M}_2,\mathbf{M}_3,\mathbf{M}_4\rbrace$ is a spanning set for the vector space $\mathbb{M}_{2\times2}$

		\begin{oneparchoices}
			\choice True \quad \choice False
		\end{oneparchoices}

		\question[1] The set $\lbrace\mathbf{M}_1,\mathbf{M}_2,\mathbf{M}_3\rbrace$ is a basis for \dots

		\begin{oneparchoices}
			\choice a subspace of $\mathbf{M}_{2\times2}$\\
			\choice the vector space $\mathbf{M}_{2\times2}$\\
			\choice cannot be a basis
		\end{oneparchoices}

		\question[1] How many different \textbf{independent} sets of TWO matrices (i.e. $\lbrace \mathbf{M}_i, \mathbf{M}_f \rbrace$) can be formed using the four given matrices?

		\begin{noparchoices}
			\choice 6~~\choice$4!$~~\choice1~~\choice3~~\choice2~~\choice$3!$	
		\end{noparchoices}
		\nextcol
		
		\newquestion{3}{Consider the following matrix\\
		$A = \ibmatrix{1&0&2&1\\0&2&1&1\\0&0&1&1\\0&0&0&4}$
		}
		\question[1] The eigenvalues of $(A^{-1})^T$ are

		\begin{oneparchoices}
			\choice $0.25, 0.5, 1$\\
			\choice $0.2, 0.25, 0.5$\\
			\choice $1,2,4$\\
			\choice $1,4,16$\\
			\choice $0,1,2,4$\\
			\choice $A$ is non-invertible
		\end{oneparchoices}

		\question[1] The eigenvalues of $(A^T)^3$ are

		\begin{oneparchoices}
			\choice $1, 2, 4$\\
			\choice $1, 8, 64$\\
			\choice $0, 0.2, 0.25, 1$\\
			\choice $0.15625, 0.125, 1$\\
			\choice $0.2, 0.25, 0.5$\\
			\choice $0, 1, 2, 4$
		\end{oneparchoices}

		\question[1] An eigenvector of $A$ corresponding to $\lambda = 4$ is

		\begin{noparchoices}
			\begin{tabular}{c l c l c l}
				\choice & $\ibmatrix{1\\2\\1\\2}$ & \choice & $\ibmatrix{1\\1\\1}$ & \choice & $\ibmatrix{2\\2\\2\\2}$\\
				\choice & $\ibmatrix{0\\1\\2\\3}$ & \choice & $\ibmatrix{2\\1\\0}$ & \choice & $\ibmatrix{1\\0\\1\\2}$
			\end{tabular}\\
		\end{noparchoices}
		\newpage
		
		\newquestion[mark(s)]{2}{}
		\question[2] Let $T : U \rightarrow V$ be a linear transformation. If $T(x,y) =\\(x+y, x - y, y)$ then $range(T) =$\linebreak

		\begin{oneparchoices}
			\choice $span\lbrace (0,1,0), (1,-1,1) \rbrace$\\
			\choice $span\lbrace (1,1,0), (0,0,1) \rbrace$\\
			\choice $\lbrace (1,1,0), (0,0,1) \rbrace$\\
			\choice $\lbrace (1,1,0), (1,-1,1) \rbrace$\\
			\choice $\lbrace (0,1,0), (1,-1,1) \rbrace$\\
			\choice $span\lbrace (1,1,0), (1,-1,1) \rbrace$
		\end{oneparchoices}

		\newquestion{}{}
		\question[1] Which of the following is a subspace of $\mathbb{R}^2$?

		\begin{tabular}{|l|l|}
			\hline
			\@Roman{1} & $\lbrace (x,y) : xy \le 0 \rbrace$\\
			\hline
			\@Roman{2} & $\lbrace (x,y) : x+y = 2 \rbrace$\\
			\hline
			\@Roman{3} & $\lbrace (x,y) : x+y \ge 0 \rbrace$\\
			\hline
			\@Roman{4} & $\lbrace (x,y) : x+y = 0 \rbrace$\\
			\hline
		\end{tabular}

		\begin{noparchoices}
			\begin{tabular}{l l}
				\choice all but \@Roman{2} & \choice Only \@Roman{3}\\
				\choice all but \@Roman{1} & \choice Only \@Roman{4}\\
				\choice only \@Roman{1} \& \@Roman{4} & \choice none\\
			\end{tabular}
		\end{noparchoices}

		\newquestion{}{}
		\question[1] Consider the matrix $A = \ibmatrix{a&b\\c&d}$ that has an eigenvalue 1 with associated eigenvector $x = \ibmatrix{2\\3}$. \linebreak
		The value of $A^{20}x$ is \dots

		\begin{noparchoices}
			{\def\arraystretch{1.2}
				\begin{tabular}{l l l l}
					\choice & $\ibmatrix{2\\3}$ & \choice & $\ibmatrix{a^{20}&b^{20}\\c^{20}&d^{20}}$ \\
					\\
					\choice & $\ibmatrix{2^{20}\\3^{20}}$ & \choice & $\ibmatrix{2\\3}^{20}$ \\
					\\
					\choice & $\ibmatrix{a&b\\c&d}$ & \choice & $\ibmatrix{a&b\\c&d}^{20}$\\
					\\
				\end{tabular}
			}
		\end{noparchoices}

		\nextcol
		
		\newquestion[mark(s)]{2}{}
		\question[2] If $T : \mathbb{R}^2 \rightarrow \mathbb{R}^3$ is a linear transformation where\\
		$T(1,0) = (2,3,1),\\
		T(1,1) = (3,0,2).$\\
		Which of the following statements is \textbf{\underline{correct}}?

		\begin{oneparchoices}
			\choice $T(x,y) = (2x - y, 3x + 3y, x - y)$\\
			\choice $T(x,y) = (x-y, 2x-y, 3x+3y)$\\
			\choice $T(x,y) = (2x+y, 3x-3y, x+y)$\\
			\choice $T(x,y) = (x+y, 2x+y, 3x-3y)$
		\end{oneparchoices}

		\newquestion{}{}
		\question[1] Which of the following is \textbf{\underline{not}} a linear transformation?

		\begin{oneparchoices}
			\choice $T : \mathbb{R} \rightarrow \mathbb{R}^2$ such that\\$T(x)=(2x,3x)$.\\
			\choice $T : \mathbb{R}^3 \rightarrow \mathbb{R}^2$ such that\\$T(x,y,z)=(z, x+y)$.\\
			\choice $T : \mathbb{R}^3 \rightarrow \mathbb{R}^2$ such that\\$T(x,y,z)=(x+1,y+z)$.\\
			\choice $T : \mathbb{R}^2 \rightarrow \mathbb{R}^2$ such that\\$T(x,y)=(2x-y,x)$.
		\end{oneparchoices}

		\newquestion[mark(s)]{2}{Consider a square matrix $A$.}
		\question[2] For a homogenous system $Ax = O$, then $|A| = 0$ means that the system\linebreak\\
		\begin{oneparchoices}
			\choice has only the zero solution\\\\
			\choice has non-zero solutions\\\\
			\choice has no solution\\\\
			\choice either has no solution or infinite number of solutions\\
		\end{oneparchoices}

		\newpage

		\question[1] For a system $Ax = B$, then $|A| = 0$ means that the system

		\begin{oneparchoices}
			\choice has infinite solutions\\
			\choice has no solution\\
			\choice has a unique solution\\
			\choice either has no solution or infinite solutions
		\end{oneparchoices}

		\newquestion{}{}
		\question[1] If matrix $\mathbf{A}$ and $\mathbf{B}$ are similar matrices, and $\mathbf{A}$ is invertible, then so is matrix $\mathbf{B}$.

		\begin{oneparchoices}
			\choice True \quad \choice False
		\end{oneparchoices}

		\newquestion[mark(s)]{2}{}
		\question[2] Consider the basis $S = \lbrace V_1, V_2, V_3 \rbrace$ for $\mathbb{R}^3$ where $V_1 = (1,1,1) \linebreak V_2 = (1,1,0)$ and $V_3 = (1,0,0)$.\linebreak
		Let $T : \mathbb{R}^3 \rightarrow \mathbb{R}^2$ be a linear transformation such that \linebreak
		$T(V_1) = (1,0), T(V_2) = (2,-1)$,\linebreak
		and $T(V_3) = (4,3).$ Then $T(2,-3,5)$ is \dots

		\begin{noparchoices}
			\begin{tabular}{l l l l}
				\choice & $(9,0)$ & \choice & $(0,0)$\\
				\\
				\choice & $(-1,5)$ & \choice & $(1,20)$\\
				\\
				\choice & $(9,23)$ & \choice & $(3,4)$\\
			\end{tabular}
		\end{noparchoices}

		\newquestion{}{}
		\question[1] The two matrices\\
		$\ibmatrix{1&0\\1&-1}$ and $\ibmatrix{0&1\\1&0}$\\
		are similar and diagonalizable.
	
		\begin{oneparchoices}
			\choice True \quad \choice False
		\end{oneparchoices}

		\nextcol

		\newquestion{}{}
		\question Let $T : x \rightarrow Ax$ be a linear transformation, where $A$ is a $3\times4$ matrix, then the vector $(1,2,3,4)$

		\begin{oneparchoices}
			\choice may belong to $range(T)$.\\\\
			\choice must belong to $range(T)$.\\\\
			\choice may belong to $kern(T)$.\\\\
			\choice must belong to $kern(T)$.\\\\
			\choice $\notin$ the domain of $T$.\\\\
			\choice $\in$ the co-domain $T$.\\
		\end{oneparchoices}

		\newpage
		\newquestion{}{}
		\question[1] For a system of linear equations whose augmented matrix in RREF is\\
		\begin{equation}
			\left \lbrack \begin{matrix}
				1 & 3 & 0 & 0 & 1\\
				0 & 0 & 1 & 0 & 2\\
				0 & 0 & 0 & 1 & 4
			\end{matrix} \right \vert
			\left . \begin{matrix}
				2 \\ 5 \\ -1
			\end{matrix} \right \rbrack \notag
		\end{equation}
		which of the following is the solution of the system? $(t,r \in \mathbb{R})$ \linebreak

		\begin{oneparchoices}
			{
				\def\arraystretch{1.1}
				\choice $\ibmatrix{2\\0\\5\\-1\\0} - t \ibmatrix{3\\1\\0\\0\\0} - r \ibmatrix{1\\0\\2\\4\\1}$\\[+1em]
				\choice $\ibmatrix{2\\0\\5\\-1} - t \ibmatrix{3\\1\\0\\0} - r \ibmatrix{1\\0\\2\\4}$\\[+1em]
				\choice $\ibmatrix{2\\0\\5\\-1} - t \ibmatrix{3\\-1\\0\\0} - r \ibmatrix{1\\0\\2\\4}$\\[+1em]
				\choice $\ibmatrix{2\\5\\-1}$\\[+1em]
				\choice $\ibmatrix{2\\0\\5\\-1\\0} - t \ibmatrix{3\\-1\\0\\0\\0} - r \ibmatrix{1\\0\\2\\4\\-1c}$\\[+1em]
				\choice $\ibmatrix{2\\5\\-1} - t \ibmatrix{3\\0\\0} - r \ibmatrix{1\\2\\4}$\\
			}
		\end{oneparchoices}

		\nextcol
		\newquestion{}{}
		\question[1] Let $A : \mathbb{R}^6 \rightarrow \mathbb{R}^5$ and $B : \mathbb{R}^5 \rightarrow \mathbb{R}^7$ be two linear transformations. Then, which of the following \textbf{\underline{can be true}}?

		\begin{oneparchoices}
			\choice $A$ and $B$ are 1-to-1.\\\\
			\choice $A$ is 1-to-1 and $B$ is not 1-to-1.\\\\
			\choice $A$ and $B$ are onto. \\\\
			\choice $A$ is 1-to-1 and $B$ is onto. \\\\
			\choice $A$ is onto and $B$ is 1-to-1. \\\\
			\choice $A$ and $B$ are isomorphic. \\
		\end{oneparchoices}

		\newquestion{}{}
		\question[1] The dimension of the standard \linebreak vector space $\mathbb{P}_3$ is \dots

		\begin{noparchoices}
			\choice 5~~\choice0~~\choice1~~\choice4~~\choice3~~\choice2 \linebreak
		\end{noparchoices}

		\endgradingrange{exam}

		\uplevel{\separator
		\fontsize{12}{14}
		Examiners' committee:
		\begin{itemize}
			\item Dr. Islam Abdul Maksoud
			\item Dr. Sara Hassan
			\item Dr. Ahmed Tayel
			\item Dr. Ahmed Yahia
		\end{itemize}
		Total marks $= \pointsinrange{exam}$
		}
	\end{questions}
\end{multicols*}
\end{document}