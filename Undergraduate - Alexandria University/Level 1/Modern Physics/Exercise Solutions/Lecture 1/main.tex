\documentclass[a4paper,12pt]{article}
\usepackage{amssymb}
\usepackage{stackrel}
\usepackage{amsmath}
\usepackage{geometry}
\usepackage{wrapfig}
\usepackage{xcolor}
\usepackage{multicol} 
\usepackage{fancyhdr}
\usepackage{lastpage}
\usepackage{tikz}
\usepackage{physics}
\usepackage{lipsum}
\usepackage{siunitx}

% Set up page layout
\geometry{
    top=1.5cm,
    bottom=1.5cm,
    left=1.5cm,
    right=1.5cm,
    headsep=0.5cm,
    headheight=3cm,
}



% Define header and footer
\pagestyle{fancy}
\fancyhf{} % Clear default header and footer
\fancyfoot[R]{\thepage/\pageref{LastPage}} % Page number at the center of the footer
\renewcommand{\headrulewidth}{0pt} % Remove header rule

\newcommand{\posaxes}{%
  \mathrel{\vbox{\offinterlineskip\ialign{%
    \hfil##\hfil\cr
    $\big\uparrow^{\text{\tiny+}}$\cr
    \noalign{\kern-1.7ex}
    $\hspace{0.38cm}\stackrel{\hspace{0.59cm}\text{\tiny+}}{\longrightarrow}$\cr
}}}}
\begin{document}

\begin{center}
\textbf{Modern Physics - Lecture Exercises Solutions\\}
\textbf{Lecture 1}


\textbf{Student Answers by Youssef Samy}
\rule{\textwidth}{0.4pt}
\end{center}


\textbf{Ex.~1} Show that although observers in $S$ and $S^{\prime}$ measure different coordinates for the ends of a stick at rest in $S$, they agree on the length of the stick.

\textbf{Sol.~1}

\begin{wrapfigure}[8]{l}{0pt}   
    \definecolor{c00aeef}{RGB}{0,174,239}
\definecolor{c231f20}{RGB}{35,31,32}
\def \globalscale {1.000000}
\begin{tikzpicture}[y=1cm, x=1cm, yscale=\globalscale,xscale=\globalscale, every node/.append style={scale=\globalscale}, inner sep=0pt, outer sep=0pt]
  \path[draw=c00aeef,miter limit=10.0] (4.7117, -3.8018) -- (4.7117, -3.7621);
  \path[draw=c00aeef,miter limit=10.0,dash pattern=on 0.0831cm off 0.0831cm] (4.7117, -3.679) -- (4.7117, -2.5569);



  \path[draw=c00aeef,miter limit=10.0] (4.7117, -2.5154) -- (4.7117, -2.4757);



  \path[draw=c00aeef,miter limit=10.0] (2.6535, -3.8018) -- (2.6535, -3.7621);



  \path[draw=c00aeef,miter limit=10.0,dash pattern=on 0.0831cm off 0.0831cm] (2.6535, -3.679) -- (2.6535, -2.5569);



  \path[draw=c00aeef,miter limit=10.0] (2.6535, -2.5154) -- (2.6535, -2.4757);



  \path[draw=c00aeef,miter limit=10.0,dash pattern=on 0.0794cm] (4.7146, -2.4757) -- (0.68, -2.4757);



  \path[draw=c231f20,fill=white,miter limit=10.0] (2.8104, -0.454) circle (0.4408cm);



  \path[draw=c231f20,miter limit=10.0] (2.8104, -2.4757) -- (2.8104, -0.9906).. controls (2.9932, -1.8272) and (3.4348, -1.733) .. (3.4348, -1.733);



  \path[draw=c231f20,miter limit=10.0] (2.8104, -2.4757) -- (2.4305, -3.32);



  \path[draw=c231f20,miter limit=10.0] (2.8104, -2.4757) -- (3.1903, -3.32);



  \path[draw=c231f20,miter limit=10.0] (2.8104, -1.0435).. controls (1.8854, -1.5478) and (2.6204, -1.701) .. (2.6204, -1.701);



  \path[draw=c231f20,miter limit=10.0] (1.8489, -0.708) -- (1.8489, -3.32);



  \node[text=c231f20,cm={ 1.26,-0.0,-0.0,1.0,(0.6535, -1.6489)},anchor=south west] (text9566) at (0.0, 0.0){$y$};



  \node[text=c231f20,cm={ 1.26,-0.0,-0.0,1.0,(1.3962, -1.5579)},anchor=south west] (text2875) at (0.0, 0.0){$S$};



  \path[draw=c231f20,miter limit=10.0] (1.8489, -0.708) -- (1.7497, -0.9893);



  \path[draw=c231f20,miter limit=10.0] (1.8489, -0.708) -- (1.9479, -0.9893);



  \path[draw=c231f20,miter limit=10.0] (4.9887, -3.32) -- (1.8489, -3.32);



  \path[draw=c231f20,miter limit=10.0] (4.9887, -3.32) -- (4.7072, -3.221);



  \path[draw=c231f20,miter limit=10.0] (4.9887, -3.32) -- (4.7072, -3.4192);



  \node[text=c231f20,cm={ 1.26,-0.0,-0.0,1.0,(6.0505, -4.5191)},anchor=south west] (text3754) at (0.0, 0.0){$x$};



  \path[draw=c231f20,fill=white,miter limit=10.0] (1.6645, -1.6489) circle (0.4408cm);



  \path[draw=c231f20,miter limit=10.0] (1.6645, -3.6706) -- (1.6645, -2.1855).. controls (1.8473, -3.0221) and (2.2889, -2.9279) .. (2.2889, -2.9279);



  \path[draw=c231f20,miter limit=10.0] (1.6645, -3.6706) -- (1.2846, -4.5149);



  \path[draw=c231f20,miter limit=10.0] (1.6645, -3.6706) -- (2.0447, -4.5149);



  \path[draw=c231f20,miter limit=10.0] (1.6645, -2.2384).. controls (0.7395, -2.7427) and (1.4745, -2.8959) .. (1.4745, -2.8959);



  \path[draw=c231f20,miter limit=10.0] (0.703, -1.9029) -- (0.703, -4.5149);



  \node[text=c231f20,cm={ 1.29,-0.0,-0.0,1.0,(1.7849, -0.5263)},anchor=south west] (text2515) at (0.0, 0.0){$y^{\prime}$};



  \node[text=c231f20,cm={ 1.26,-0.0,-0.0,1.0,(1.0618, -1.0435)},anchor=south west] (text9280) at (0.0, 0.0){$S^{\prime}$};



  \path[draw=c231f20,miter limit=10.0] (0.703, -1.9029) -- (0.604, -2.1841);



  \path[draw=c231f20,miter limit=10.0] (0.703, -1.9029) -- (0.8022, -2.1841);



  \path[draw=c231f20,miter limit=10.0] (5.8997, -4.5191) -- (0.6842, -4.5191);



  \path[draw=c231f20,miter limit=10.0] (5.8941, -4.5191) -- (5.6129, -4.4199);



  \path[draw=c231f20,miter limit=10.0] (5.8941, -4.5191) -- (5.6129, -4.618);



  \node[text=c231f20,cm={ 1.26,-0.0,-0.0,1.0,(5.2515, -3.3695)},anchor=south west] (text4730) at (0.0, 0.0){$x^\prime$};



  \path[draw=c231f20,miter limit=10.0] (2.6535, -3.8018) -- (4.7117, -3.8018);



  \path[draw=c231f20,miter limit=10.0] (2.6535, -3.8018) -- (2.6535, -3.9415);



  \path[draw=c231f20,miter limit=10.0] (2.6535, -3.9415) -- (4.7117, -3.9415);



  \path[draw=c231f20,miter limit=10.0] (4.7117, -3.8018) -- (4.7117, -3.9415);



  \node[text=c231f20,cm={ 1.26,-0.0,-0.0,1.0,(0.0, -3.0311)},anchor=south west] (text2968) at (0.0, 0.0){$S$};



  \path[draw=c231f20,miter limit=10.0] (2.6535, -3.9415) -- (2.6535, -3.9812);



  \path[draw=c231f20,miter limit=10.0,dash pattern=on 0.0701cm off 0.0701cm] (2.6535, -4.0513) -- (2.6535, -4.1566);



  \path[draw=c231f20,miter limit=10.0] (2.6535, -4.1915) -- (2.6535, -4.2312);



  \path[draw=c231f20,miter limit=10.0] (4.7117, -3.9415) -- (4.7088, -3.9809);



  \path[draw=c231f20,miter limit=10.0,dash pattern=on 0.0693cm off 0.0693cm] (4.704, -4.0502) -- (4.6966, -4.154);



  \path[draw=c231f20,miter limit=10.0] (4.6942, -4.1886) -- (4.6916, -4.228);



  \node[text=c231f20,cm={ 1.26,-0.0,-0.0,1.0,(0.3601, -4.672)},anchor=south west] (text9964) at (0.0, 0.0){O};



  \node[text=c231f20,cm={ 1.26,-0.0,-0.0,1.0,(1.8547, -3.2705)},anchor=south west] (text2136) at (0.0, 0.0){O$^\prime$};



  \node[text=c231f20,cm={ 1.26,-0.0,-0.0,1.0,(2.5247, -4.4257)},anchor=south west] (text19) at (0.0, 0.0){$x_1$};



  \node[text=c00aeef,cm={ 1.26,-0.0,-0.0,1.0,(2.3696, -2.3421)},anchor=south west] (text8886) at (0.0, 0.0){$x_1^\prime$};



  \node[text=c231f20,cm={ 1.26,-0.0,-0.0,1.0,(4.4902, -4.4222)},anchor=south west] (text5227) at (0.0, 0.0){$x_2$};



  \node[text=c00aeef,cm={ 1.26,-0.0,-0.0,1.0,(4.4196, -2.3442)},anchor=south west] (text8645) at (0.0, 0.0){$x_2^\prime$};



  \path[draw=c231f20,miter limit=10.0,dash pattern=on 0.0794cm] (4.6916, -4.228) -- (0.68, -4.228);



  \path[fill=c231f20] (1.7283, -1.0951).. controls (1.7542, -1.0906) and (1.7801, -1.0877) .. (1.806, -1.084).. controls (1.8124, -1.0832) and (1.819, -1.0819) .. (1.8254, -1.0816).. controls (1.8317, -1.0816) and (1.8383, -1.0821) .. (1.8447, -1.0819) -- (2.3498, -1.0819) -- (2.3498, -1.1083) -- (1.8447, -1.1083).. controls (1.8383, -1.1083) and (1.8317, -1.1083) .. (1.8254, -1.1086).. controls (1.819, -1.1083) and (1.8124, -1.107) .. (1.806, -1.1062).. controls (1.7801, -1.1025) and (1.7542, -1.0996) .. (1.7283, -1.0951) -- cycle;



  \path[fill=c231f20] (2.4728, -1.0951).. controls (2.4149, -1.1165) and (2.3432, -1.1533) .. (2.2987, -1.1919) -- (2.3336, -1.0951) -- (2.2987, -0.9983).. controls (2.3432, -1.0372) and (2.4149, -1.0737) .. (2.4728, -1.0951) -- cycle;



  \node[text=c231f20,cm={ 1.26,-0.0,-0.0,1.0,(2.0648, -0.8948)},anchor=south west] (text3953) at (0.0, 0.0){$v$};



  \node[text=c00aeef,cm={ 1.35,-0.0,-0.0,1.0,(0.9485, -2.3421)},anchor=south west] (text8805) at (0.0, 0.0){$vt$};
\end{tikzpicture}
\end{wrapfigure}

$\bullet$ Known:
\begin{align}
    \text{System $S$ is stationary} \tag{a}\\
    \text{System $S^\prime$ is moving with velocity $v$ relative to $S$} \tag{b} \\
    L = x_2 - x_1 \tag{c} \\
    L^\prime = x_2^\prime - x_1^\prime \tag{d}
\end{align}

$\bullet$ Showing that $L = L^\prime$ using Galilean Transformation:
\begin{align}
    x_1^\prime &= x_1 - vt \\
    x_2^\prime &= x_2 - vt \\
    L^\prime &= x_2^\prime - x_1^\prime \tag{d} \\
    L^\prime &= (x_2 - vt) - (x_1 - vt) \notag \\
    L^\prime &= x_2 - x_1 \\
    \therefore L^\prime &= L \tag{from (c), (d), and (3)}
\end{align}
\begin{center}
    \rule{6cm}{0.4pt}    
\end{center}

\textbf{Ex.~2} \textit{Conservation of Linear Momentum Is Covariant Under the
Galilean Transformation.}

Assume that two masses $m_1$ and $m_2$ are
moving in the positive $x$ direction with velocities $v_1$ and $v_2$ as
measured by an observer in $S$ before a collision. After the collision,
the two masses stick together and move with a velocity $v$ in $S$.
Show that if an observer in $S$ finds momentum to be conserved, so
does an observer in $S^\prime$.

\textbf{Sol.~2}

\begin{wrapfigure}[6]{l}{0pt}   
    \definecolor{c231f20}{RGB}{35,31,32}

\def \globalscale {1.000000}
\begin{tikzpicture}[y=1cm, x=1cm, yscale=\globalscale,xscale=\globalscale, every node/.append style={scale=\globalscale}, inner sep=0pt, outer sep=0pt]
  \path[draw=c231f20,miter limit=10.0] (1.9029, -0.417) -- (1.9029, -3.029);



  \node[text=c231f20,cm={ 1.26,-0.0,-0.0,1.0,(0.4958, -1.8209)},anchor=south west] (text9035) at (0.0, 0.0){$y$};



  \path[draw=c231f20,miter limit=10.0] (1.9029, -0.417) -- (1.8039, -0.6982);



  \path[draw=c231f20,miter limit=10.0] (1.9029, -0.417) -- (2.0021, -0.6982);



  \path[draw=c231f20,miter limit=10.0] (5.043, -3.029) -- (1.9029, -3.029);



  \path[draw=c231f20,miter limit=10.0] (5.043, -3.029) -- (4.7614, -2.93);



  \path[draw=c231f20,miter limit=10.0] (5.043, -3.029) -- (4.7614, -3.1282);



  \node[text=c231f20,cm={ 1.26,-0.0,-0.0,1.0,(5.8931, -4.6911)},anchor=south west] (text5310) at (0.0, 0.0){$x$};



  \path[draw=c231f20,miter limit=10.0] (0.5453, -2.0749) -- (0.5453, -4.6871);



  \node[text=c231f20,cm={ 1.29,-0.0,-0.0,1.0,(1.8391, -0.2352)},anchor=south west] (text733) at (0.0, 0.0){$y^\prime$};



  \node[text=c231f20,cm={ 1.26,-0.0,-0.0,1.0,(1.365, -1.9444)},anchor=south west] (text7650) at (0.0, 0.0){$S^\prime$};



  \path[draw=c231f20,miter limit=10.0] (0.5453, -2.0749) -- (0.4464, -2.3561);



  \path[draw=c231f20,miter limit=10.0] (0.5453, -2.0749) -- (0.6445, -2.3561);



  \path[draw=c231f20,miter limit=10.0] (5.742, -4.6911) -- (0.5265, -4.6911);



  \path[draw=c231f20,miter limit=10.0] (5.7364, -4.6911) -- (5.4552, -4.5921);



  \path[draw=c231f20,miter limit=10.0] (5.7364, -4.6911) -- (5.4552, -4.79);



  \node[text=c231f20,cm={ 1.26,-0.0,-0.0,1.0,(5.1525, -2.8623)},anchor=south west] (text4647) at (0.0, 0.0){$x^\prime$};



  \node[text=c231f20,cm={ 1.26,-0.0,-0.0,1.0,(0.1341, -3.0361)},anchor=south west] (text989) at (0.0, 0.0){$S$};



  \node[text=c231f20,cm={ 1.26,-0.0,-0.0,1.0,(0.0, -4.9427)},anchor=south west] (text5260) at (0.0, 0.0){O};



  \path[fill=c231f20] (1.7822, -0.8041).. controls (1.8082, -0.7996) and (1.8341, -0.7967) .. (1.86, -0.793).. controls (1.8664, -0.7922) and (1.873, -0.7908) .. (1.8793, -0.7906).. controls (1.8857, -0.7906) and (1.8923, -0.7911) .. (1.8987, -0.7908) -- (2.4037, -0.7908) -- (2.4037, -0.8173) -- (1.8987, -0.8173).. controls (1.8923, -0.8173) and (1.8857, -0.8173) .. (1.8793, -0.8176).. controls (1.873, -0.8173) and (1.8664, -0.816) .. (1.86, -0.8152).. controls (1.8341, -0.8115) and (1.8082, -0.8086) .. (1.7822, -0.8041) -- cycle;



  \path[fill=c231f20] (2.527, -0.8041).. controls (2.4691, -0.8255) and (2.3974, -0.8623) .. (2.3529, -0.9009) -- (2.3879, -0.8041) -- (2.3529, -0.7072).. controls (2.3974, -0.7461) and (2.4691, -0.7826) .. (2.527, -0.8041) -- cycle;



  \node[text=c231f20,cm={ 1.26,-0.0,-0.0,1.0,(2.119, -0.6038)},anchor=south west] (text256) at (0.0, 0.0){$u$};



  \node[text=c231f20,cm={ 1.26,-0.0,-0.0,1.0,(1.4258, -3.1231)},anchor=south west] (text8061) at (0.0, 0.0){O$^\prime$};



  \path[draw=c231f20,miter limit=10.0] (1.5899, -3.9426) circle (0.4109cm);



  \path[draw=c231f20,miter limit=10.0] (3.3457, -3.9672) circle (0.4109cm);



  \path[fill=c231f20] (1.2176, -4.4773).. controls (1.2435, -4.4728) and (1.2695, -4.4699) .. (1.2954, -4.4662).. controls (1.3018, -4.4654) and (1.3084, -4.4641) .. (1.3147, -4.4638).. controls (1.3211, -4.4638) and (1.3277, -4.4643) .. (1.334, -4.4641) -- (1.8391, -4.4641) -- (1.8391, -4.4905) -- (1.334, -4.4905).. controls (1.3277, -4.4905) and (1.3211, -4.4905) .. (1.3147, -4.4908).. controls (1.3084, -4.4905) and (1.3017, -4.4892) .. (1.2954, -4.4884).. controls (1.2695, -4.4847) and (1.2435, -4.4818) .. (1.2176, -4.4773) -- cycle;



  \path[fill=c231f20] (1.9624, -4.4773).. controls (1.9045, -4.4987) and (1.8328, -4.5355) .. (1.7883, -4.5741) -- (1.8232, -4.4773) -- (1.7883, -4.3804).. controls (1.8328, -4.4193) and (1.9045, -4.4558) .. (1.9624, -4.4773) -- cycle;



  \node[text=c231f20,cm={ 1.26,-0.0,-0.0,1.0,(2.001, -4.5268)},anchor=south west] (text588) at (0.0, 0.0){$v_1$};



  \path[draw=c231f20,miter limit=10.0] (1.1102, -3.0689).. controls (1.0814, -3.0528) and (1.0268, -3.0279) .. (0.9583, -3.0319).. controls (0.8559, -3.0377) and (0.7461, -3.1059) .. (0.7509, -3.162).. controls (0.7551, -3.2112) and (0.8464, -3.2422) .. (0.9131, -3.265).. controls (0.9776, -3.2869) and (1.0351, -3.2935) .. (1.0766, -3.2954);



  \path[draw=c231f20,fill=c231f20,miter limit=10.0] (0.893, -3.0361).. controls (0.9136, -3.0528) and (0.9514, -3.088) .. (0.9578, -3.1366).. controls (0.9655, -3.1951) and (0.9237, -3.2385) .. (0.9046, -3.2557);



  \path[draw=c231f20,miter limit=10.0] (2.4445, -2.1796).. controls (2.4156, -2.1635) and (2.3611, -2.1386) .. (2.2926, -2.1426).. controls (2.1902, -2.1484) and (2.0804, -2.2167) .. (2.0852, -2.2728).. controls (2.0894, -2.322) and (2.1807, -2.3529) .. (2.2474, -2.3757).. controls (2.3119, -2.3977) and (2.3693, -2.4043) .. (2.4109, -2.4061);



  \path[draw=c231f20,fill=c231f20,miter limit=10.0] (2.227, -2.1468).. controls (2.2476, -2.1635) and (2.2855, -2.1987) .. (2.2918, -2.2474).. controls (2.2995, -2.3058) and (2.2577, -2.3492) .. (2.2386, -2.3664);



  \path[draw=c231f20,miter limit=10.0,dash pattern=on 0.0794cm] (4.6133, -3.7999) circle (0.5048cm);



  \path[fill=c231f20] (4.2508, -4.4196).. controls (4.2773, -4.4151) and (4.3037, -4.4119) .. (4.3302, -4.4082) -- (4.3302, -4.431).. controls (4.3037, -4.4273) and (4.2773, -4.4241) .. (4.2508, -4.4196) -- cycle;



  \path[fill=c231f20] (4.4095, -4.4064) rectangle (4.4889, -4.4328);



  \path[fill=c231f20] (4.5683, -4.4064) rectangle (4.6477, -4.4328);



  \path[fill=c231f20] (4.727, -4.4064) rectangle (4.8064, -4.4328);



  \path[fill=c231f20] (4.9953, -4.4196).. controls (4.9374, -4.441) and (4.8657, -4.4778) .. (4.8212, -4.5164) -- (4.8562, -4.4196) -- (4.8212, -4.3228).. controls (4.8657, -4.3617) and (4.9374, -4.3982) .. (4.9953, -4.4196) -- cycle;



  \node[text=c231f20,cm={ 1.26,-0.0,-0.0,1.0,(5.0604, -4.4773)},anchor=south west] (text3774) at (0.0, 0.0){$v$};



  \node[text=c231f20,cm={ 1.26,-0.0,-0.0,1.0,(1.2907, -4.151)},anchor=south west] (text6657) at (0.0, 0.0){$m_1$};



  \node[text=c231f20,cm={ 1.26,-0.0,-0.0,1.0,(3.0044, -4.1974)},anchor=south west] (text5195) at (0.0, 0.0){$m_2$};



  \node[text=c231f20,cm={ 1.26,-0.0,-0.0,1.0,(4.2508, -4.0)},anchor=south west] (text5524) at (0.0, 0.0){$m_t$};

  
  \path[fill=c231f20] (3.7182, -4.5241).. controls (3.6923, -4.5286) and (3.6663, -4.5315) .. (3.6404, -4.5352).. controls (3.6341, -4.536) and (3.6274, -4.5373) .. (3.6211, -4.5376).. controls (3.6147, -4.5376) and (3.6081, -4.5371) .. (3.6018, -4.5373) -- (3.0967, -4.5373) -- (3.0967, -4.5109) -- (3.6018, -4.5109).. controls (3.6081, -4.5109) and (3.6147, -4.5109) .. (3.6211, -4.5106).. controls (3.6274, -4.5109) and (3.6341, -4.5122) .. (3.6404, -4.513).. controls (3.6663, -4.5167) and (3.6923, -4.5196) .. (3.7182, -4.5241) -- cycle;



  \path[fill=c231f20] (2.9734, -4.5241).. controls (3.0313, -4.5027) and (3.103, -4.4659) .. (3.1475, -4.4273) -- (3.1126, -4.5241) -- (3.1475, -4.6209).. controls (3.103, -4.5821) and (3.0313, -4.5455) .. (2.9734, -4.5241) -- cycle;

  \node[text=c231f20,cm={ 1.26,-0.0,-0.0,1.0,(2.6088, -4.4183)},anchor=south west] (text1977) at (0.0, 0.0){$v_2$};


\end{tikzpicture}


\end{wrapfigure}
\noindent We perform Galilean transformation.\\
$\bullet$ Known:
\begin{align}
    \text{System $S$ is stationary} \tag{a}\\
    \text{System $S^\prime$ is moving with velocity $u$ relative to $S$} \tag{b} \\
    m_1 = m_1^\prime,  \quad m_2 = m_2^\prime \text{~(invariance of mass)} \tag{c} \\
    \text{Motion of observer in $S^\prime$ is in the positive $x$-direction} \tag{d} \\
    \text{Direction of positive velocity} \posaxes \tag{e} \\
    v_{\text{event}}^\prime = v_{\text{event}} - u \text{~for all events (covariance of velocity)} \tag{f} \\
    m_1 v_1 + m_2 v_2 = (m_1+m_2) v \text{~(conserved momentum)} \tag{g}
\end{align}
\ \\
$\bullet$ Showing that the momentum is conserved under $S^\prime$:
\begin{align}
m_1^\prime v_1^\prime + m_2^\prime v_2^\prime &= (m_1^\prime + m_2^\prime) v^\prime \tag{to be proven} \\
m_1 (v_1 - u) + m_2 (v_2 - u) &= (m_1 + m_2) (v-u) \text{~(Galilean transformation)} \tag{1} \\
m_1 v_1 - \underline{m_1 u} + m_2 v_2 - \underline{m_2 u} &= m_1 v + m_2 v - \underline{m_1 u} - \underline{m_2 u} \tag{2} \\
m_1 v_1 + m_2 v_2 &= (m_1 + m_2) v \tag{3} \\
\therefore \text{Momentum is conserved} \tag{from (g) and (3)}
\end{align}
\begin{center}
    \rule{6cm}{0.4pt}    
\end{center}
\newpage
\textbf{Ex.~3} A 2000-kg car moving with a speed of 20 m/s collides with and sticks to a 1500-kg car at rest at a stop sign. Show that because momentum is conserved in the rest frame, momentum is also conserved in a reference frame moving with a speed of 10 m/s in the direction of the moving car.

\textbf{Sol.~3}

\begin{wrapfigure}[6]{l}{6cm}
    \definecolor{c231f20}{RGB}{35,31,32}

\def \globalscale {1.000000}
\begin{tikzpicture}[y=1cm, x=1cm, yscale=\globalscale,xscale=\globalscale, every node/.append style={scale=\globalscale}, inner sep=0pt, outer sep=0pt]
  \path[draw=c231f20,miter limit=10.0] (1.9029, -0.417) -- (1.9029, -3.029);



  \node[text=c231f20,cm={ 1.26,-0.0,-0.0,1.0,(0.4958, -1.8209)},anchor=south west] (text9035) at (0.0, 0.0){$y$};



  \path[draw=c231f20,miter limit=10.0] (1.9029, -0.417) -- (1.8039, -0.6982);



  \path[draw=c231f20,miter limit=10.0] (1.9029, -0.417) -- (2.0021, -0.6982);



  \path[draw=c231f20,miter limit=10.0] (5.043, -3.029) -- (1.9029, -3.029);



  \path[draw=c231f20,miter limit=10.0] (5.043, -3.029) -- (4.7614, -2.93);



  \path[draw=c231f20,miter limit=10.0] (5.043, -3.029) -- (4.7614, -3.1282);



  \node[text=c231f20,cm={ 1.26,-0.0,-0.0,1.0,(5.8931, -4.6911)},anchor=south west] (text5310) at (0.0, 0.0){$x$};



  \path[draw=c231f20,miter limit=10.0] (0.5453, -2.0749) -- (0.5453, -4.6871);



  \node[text=c231f20,cm={ 1.29,-0.0,-0.0,1.0,(1.8391, -0.2352)},anchor=south west] (text733) at (0.0, 0.0){$y^\prime$};



  \node[text=c231f20,cm={ 1.26,-0.0,-0.0,1.0,(1.365, -1.9444)},anchor=south west] (text7650) at (0.0, 0.0){$S^\prime$};



  \path[draw=c231f20,miter limit=10.0] (0.5453, -2.0749) -- (0.4464, -2.3561);



  \path[draw=c231f20,miter limit=10.0] (0.5453, -2.0749) -- (0.6445, -2.3561);



  \path[draw=c231f20,miter limit=10.0] (5.742, -4.6911) -- (0.5265, -4.6911);



  \path[draw=c231f20,miter limit=10.0] (5.7364, -4.6911) -- (5.4552, -4.5921);



  \path[draw=c231f20,miter limit=10.0] (5.7364, -4.6911) -- (5.4552, -4.79);



  \node[text=c231f20,cm={ 1.26,-0.0,-0.0,1.0,(5.1525, -2.8623)},anchor=south west] (text4647) at (0.0, 0.0){$x^\prime$};



  \node[text=c231f20,cm={ 1.26,-0.0,-0.0,1.0,(0.1341, -3.0361)},anchor=south west] (text989) at (0.0, 0.0){$S$};



  \node[text=c231f20,cm={ 1.26,-0.0,-0.0,1.0,(0.0, -4.9427)},anchor=south west] (text5260) at (0.0, 0.0){O};



  \path[fill=c231f20] (1.7822, -0.8041).. controls (1.8082, -0.7996) and (1.8341, -0.7967) .. (1.86, -0.793).. controls (1.8664, -0.7922) and (1.873, -0.7908) .. (1.8793, -0.7906).. controls (1.8857, -0.7906) and (1.8923, -0.7911) .. (1.8987, -0.7908) -- (2.4037, -0.7908) -- (2.4037, -0.8173) -- (1.8987, -0.8173).. controls (1.8923, -0.8173) and (1.8857, -0.8173) .. (1.8793, -0.8176).. controls (1.873, -0.8173) and (1.8664, -0.816) .. (1.86, -0.8152).. controls (1.8341, -0.8115) and (1.8082, -0.8086) .. (1.7822, -0.8041) -- cycle;



  \path[fill=c231f20] (2.527, -0.8041).. controls (2.4691, -0.8255) and (2.3974, -0.8623) .. (2.3529, -0.9009) -- (2.3879, -0.8041) -- (2.3529, -0.7072).. controls (2.3974, -0.7461) and (2.4691, -0.7826) .. (2.527, -0.8041) -- cycle;



  \node[text=c231f20,cm={ 1.26,-0.0,-0.0,1.0,(2.119, -0.6038)},anchor=south west] (text256) at (0.0, 0.0){$u$};



  \node[text=c231f20,cm={ 1.26,-0.0,-0.0,1.0,(1.4258, -3.1231)},anchor=south west] (text8061) at (0.0, 0.0){O$^\prime$};



  \path[draw=c231f20,miter limit=10.0] (1.5899, -3.9426) circle (0.4109cm);



  \path[draw=c231f20,miter limit=10.0] (3.3457, -3.9672) circle (0.4109cm);



  \path[fill=c231f20] (1.2176, -4.4773).. controls (1.2435, -4.4728) and (1.2695, -4.4699) .. (1.2954, -4.4662).. controls (1.3018, -4.4654) and (1.3084, -4.4641) .. (1.3147, -4.4638).. controls (1.3211, -4.4638) and (1.3277, -4.4643) .. (1.334, -4.4641) -- (1.8391, -4.4641) -- (1.8391, -4.4905) -- (1.334, -4.4905).. controls (1.3277, -4.4905) and (1.3211, -4.4905) .. (1.3147, -4.4908).. controls (1.3084, -4.4905) and (1.3017, -4.4892) .. (1.2954, -4.4884).. controls (1.2695, -4.4847) and (1.2435, -4.4818) .. (1.2176, -4.4773) -- cycle;



  \path[fill=c231f20] (1.9624, -4.4773).. controls (1.9045, -4.4987) and (1.8328, -4.5355) .. (1.7883, -4.5741) -- (1.8232, -4.4773) -- (1.7883, -4.3804).. controls (1.8328, -4.4193) and (1.9045, -4.4558) .. (1.9624, -4.4773) -- cycle;



  \node[text=c231f20,cm={ 1.26,-0.0,-0.0,1.0,(2.001, -4.5268)},anchor=south west] (text588) at (0.0, 0.0){$v_1$};



  \path[draw=c231f20,miter limit=10.0] (1.1102, -3.0689).. controls (1.0814, -3.0528) and (1.0268, -3.0279) .. (0.9583, -3.0319).. controls (0.8559, -3.0377) and (0.7461, -3.1059) .. (0.7509, -3.162).. controls (0.7551, -3.2112) and (0.8464, -3.2422) .. (0.9131, -3.265).. controls (0.9776, -3.2869) and (1.0351, -3.2935) .. (1.0766, -3.2954);



  \path[draw=c231f20,fill=c231f20,miter limit=10.0] (0.893, -3.0361).. controls (0.9136, -3.0528) and (0.9514, -3.088) .. (0.9578, -3.1366).. controls (0.9655, -3.1951) and (0.9237, -3.2385) .. (0.9046, -3.2557);



  \path[draw=c231f20,miter limit=10.0] (2.4445, -2.1796).. controls (2.4156, -2.1635) and (2.3611, -2.1386) .. (2.2926, -2.1426).. controls (2.1902, -2.1484) and (2.0804, -2.2167) .. (2.0852, -2.2728).. controls (2.0894, -2.322) and (2.1807, -2.3529) .. (2.2474, -2.3757).. controls (2.3119, -2.3977) and (2.3693, -2.4043) .. (2.4109, -2.4061);



  \path[draw=c231f20,fill=c231f20,miter limit=10.0] (2.227, -2.1468).. controls (2.2476, -2.1635) and (2.2855, -2.1987) .. (2.2918, -2.2474).. controls (2.2995, -2.3058) and (2.2577, -2.3492) .. (2.2386, -2.3664);



  \path[draw=c231f20,miter limit=10.0,dash pattern=on 0.0794cm] (4.6133, -3.7999) circle (0.5048cm);



  \path[fill=c231f20] (4.2508, -4.4196).. controls (4.2773, -4.4151) and (4.3037, -4.4119) .. (4.3302, -4.4082) -- (4.3302, -4.431).. controls (4.3037, -4.4273) and (4.2773, -4.4241) .. (4.2508, -4.4196) -- cycle;



  \path[fill=c231f20] (4.4095, -4.4064) rectangle (4.4889, -4.4328);



  \path[fill=c231f20] (4.5683, -4.4064) rectangle (4.6477, -4.4328);



  \path[fill=c231f20] (4.727, -4.4064) rectangle (4.8064, -4.4328);



  \path[fill=c231f20] (4.9953, -4.4196).. controls (4.9374, -4.441) and (4.8657, -4.4778) .. (4.8212, -4.5164) -- (4.8562, -4.4196) -- (4.8212, -4.3228).. controls (4.8657, -4.3617) and (4.9374, -4.3982) .. (4.9953, -4.4196) -- cycle;



  \node[text=c231f20,cm={ 1.26,-0.0,-0.0,1.0,(5.0604, -4.4773)},anchor=south west] (text3774) at (0.0, 0.0){$v$};



  \node[text=c231f20,cm={ 1.26,-0.0,-0.0,1.0,(1.2907, -4.151)},anchor=south west] (text6657) at (0.0, 0.0){$m_1$};



  \node[text=c231f20,cm={ 1.26,-0.0,-0.0,1.0,(3.0044, -4.1974)},anchor=south west] (text5195) at (0.0, 0.0){$m_2$};



  \node[text=c231f20,cm={ 1.26,-0.0,-0.0,1.0,(4.2508, -4.0)},anchor=south west] (text5524) at (0.0, 0.0){$m_t$};


\end{tikzpicture}


\end{wrapfigure}
We perform Galilean transformation.

$\bullet$ Known:
\begin{align}
    \text{System $S$ is stationary} \tag{a}\\
    \text{System $S^\prime$ is moving with velocity $u=10 \si[per-mode=symbol]{\meter \per \second}$ relative to $S$} \tag{b} \\
    m_1 = m_1^\prime,  \quad m_2 = m_2^\prime \text{~(invariance of mass)} \tag{c} \\
    \text{Motion of observer in $S^\prime$ is in the positive $x$-direction} \tag{d} \\
    \text{Direction of positive velocity} \posaxes \tag{e} \\
    v_{\text{event}}^\prime = v_{\text{event}} - u \text{~for all events (covariance of velocity)} \tag{f} \\
    m_1 v_1 + m_2 v_2 = (m_1+m_2) v \text{~(conserved momentum)} \tag{g} \\
    m_1 = 2000 \si{\kilogram}, m_2 = 1500 \si{\kilogram}, v_1 = 20 \si[per-mode=symbol]{\meter \per \second}, v_2 = 0 \si[per-mode=symbol]{\meter \per \second} \tag{h} \\
    2000 \times 20 + 1500 \times 0 = 3500 \times v \tag{i} \\
    v \approx 11.43 \text{~up to the nearest $2^{nd}$ decimal point} \tag{j}
\end{align}
\ \\
$\bullet$ Showing that the momentum is conserved under $S^\prime$:
\begin{align}
m_1^\prime v_1^\prime + m_2^\prime v_2^\prime &= (m_1^\prime + m_2^\prime) v^\prime \tag{to be proven} \\
m_1 (v_1 - u) + m_2 (v_2 - u)  &= (m_1 + m_2) (v-u) \text{~(Galilean transformation)} \tag{1} \\
2000 \times (20 - 10) + 1500 \times (-10) &= 3500 \times (11.43 - 10) \tag{2} \\
\text{Left hand side} &= 5000 \tag{3} \\
\text{Right hand side} &= 5005 \tag{4} \\
\text{Left hand side} &\approx \text{Right hand side} \tag{5} \\
\therefore \text{Momentum is conserved} \tag{from (5)}
\end{align}
\begin{center}
    \rule{6cm}{0.4pt}    
\end{center}

\textbf{Ex.~4} A billiard ball of mass 0.3 kg moves with a speed of 5 m/s and collides elastically with a ball of mass 0.2 kg moving in the opposite direction with a speed of 3 m/s. Show that because momentum is conserved in the rest frame, it is also conserved in a frame of reference moving with a speed of 2 m/s in the direction of the second ball.

\textbf{Sol.~4}

\begin{wrapfigure}[5]{l}{6cm}
    
\definecolor{c231f20}{RGB}{35,31,32}


\def \globalscale {1.000000}
\begin{tikzpicture}[y=1cm, x=1cm, yscale=\globalscale,xscale=\globalscale, every node/.append style={scale=\globalscale}, inner sep=0pt, outer sep=0pt]
  \path[draw=c231f20,miter limit=10.0] (1.9029, -0.417) -- (1.9029, -3.029);



  \node[text=c231f20,cm={ 1.26,-0.0,-0.0,1.0,(0.4958, -1.8209)},anchor=south west] (text559) at (0.0, 0.0){$y$};



  \path[draw=c231f20,miter limit=10.0] (1.9029, -0.417) -- (1.8039, -0.6982);



  \path[draw=c231f20,miter limit=10.0] (1.9029, -0.417) -- (2.0021, -0.6982);



  \path[draw=c231f20,miter limit=10.0] (5.043, -3.029) -- (1.9029, -3.029);



  \path[draw=c231f20,miter limit=10.0] (5.043, -3.029) -- (4.7614, -2.93);



  \path[draw=c231f20,miter limit=10.0] (5.043, -3.029) -- (4.7614, -3.1282);



  \node[text=c231f20,cm={ 1.26,-0.0,-0.0,1.0,(5.8931, -4.6911)},anchor=south west] (text4883) at (0.0, 0.0){$x$};



  \path[draw=c231f20,miter limit=10.0] (0.5453, -2.0749) -- (0.5453, -4.6871);



  \node[text=c231f20,cm={ 1.29,-0.0,-0.0,1.0,(1.8391, -0.2352)},anchor=south west] (text3180) at (0.0, 0.0){$y^\prime$};



  \node[text=c231f20,cm={ 1.26,-0.0,-0.0,1.0,(1.365, -1.9444)},anchor=south west] (text7725) at (0.0, 0.0){$S^\prime$};



  \path[draw=c231f20,miter limit=10.0] (0.5453, -2.0749) -- (0.4464, -2.3561);



  \path[draw=c231f20,miter limit=10.0] (0.5453, -2.0749) -- (0.6445, -2.3561);



  \path[draw=c231f20,miter limit=10.0] (5.742, -4.6911) -- (0.5265, -4.6911);



  \path[draw=c231f20,miter limit=10.0] (5.7364, -4.6911) -- (5.4552, -4.5921);



  \path[draw=c231f20,miter limit=10.0] (5.7364, -4.6911) -- (5.4552, -4.79);



  \node[text=c231f20,cm={ 1.26,-0.0,-0.0,1.0,(0.1341, -3.0361)},anchor=south west] (text9186) at (0.0, 0.0){$S$};



  \node[text=c231f20,cm={ 1.26,-0.0,-0.0,1.0,(0.0, -4.9427)},anchor=south west] (text3075) at (0.0, 0.0){O};



  \path[fill=c231f20] (1.7822, -0.8041).. controls (1.8082, -0.7996) and (1.8341, -0.7967) .. (1.86, -0.793).. controls (1.8664, -0.7922) and (1.873, -0.7908) .. (1.8793, -0.7906).. controls (1.8857, -0.7906) and (1.8923, -0.7911) .. (1.8987, -0.7908) -- (2.4037, -0.7908) -- (2.4037, -0.8173) -- (1.8987, -0.8173).. controls (1.8923, -0.8173) and (1.8857, -0.8173) .. (1.8793, -0.8176).. controls (1.873, -0.8173) and (1.8664, -0.816) .. (1.86, -0.8152).. controls (1.8341, -0.8115) and (1.8082, -0.8086) .. (1.7822, -0.8041) -- cycle;



  \path[fill=c231f20] (2.527, -0.8041).. controls (2.4691, -0.8255) and (2.3974, -0.8623) .. (2.3529, -0.9009) -- (2.3879, -0.8041) -- (2.3529, -0.7072).. controls (2.3974, -0.7461) and (2.4691, -0.7826) .. (2.527, -0.8041) -- cycle;

  \node[text=c231f20,cm={ 1.26,-0.0,-0.0,1.0,(5.1525, -2.8623)},anchor=south west] (text4647) at (0.0, 0.0){$x^\prime$};



  \node[text=c231f20,cm={ 1.26,-0.0,-0.0,1.0,(2.119, -0.6038)},anchor=south west] (text6552) at (0.0, 0.0){$u$};



  \node[text=c231f20,cm={ 1.26,-0.0,-0.0,1.0,(1.37, -3.1231)},anchor=south west] (text8355) at (0.0, 0.0){O$^\prime$};



  \path[draw=c231f20,miter limit=10.0] (1.5899, -3.9426) circle (0.4109cm);



  \path[draw=c231f20,miter limit=10.0] (3.3457, -3.9672) circle (0.4109cm);



  \path[fill=c231f20] (1.2176, -4.4773).. controls (1.2435, -4.4728) and (1.2695, -4.4699) .. (1.2954, -4.4662).. controls (1.3018, -4.4654) and (1.3084, -4.4641) .. (1.3147, -4.4638).. controls (1.3211, -4.4638) and (1.3277, -4.4643) .. (1.334, -4.4641) -- (1.8391, -4.4641) -- (1.8391, -4.4905) -- (1.334, -4.4905).. controls (1.3277, -4.4905) and (1.3211, -4.4905) .. (1.3147, -4.4908).. controls (1.3084, -4.4905) and (1.3017, -4.4892) .. (1.2954, -4.4884).. controls (1.2695, -4.4847) and (1.2435, -4.4818) .. (1.2176, -4.4773) -- cycle;



  \path[fill=c231f20] (1.9624, -4.4773).. controls (1.9045, -4.4987) and (1.8328, -4.5355) .. (1.7883, -4.5741) -- (1.8232, -4.4773) -- (1.7883, -4.3804).. controls (1.8328, -4.4193) and (1.9045, -4.4558) .. (1.9624, -4.4773) -- cycle;



  \node[text=c231f20,cm={ 1.26,-0.0,-0.0,1.0,(1.302, -4.102)},anchor=south west] (text5908) at (0.0, 0.0){$m_1$};



  \node[text=c231f20,cm={ 1.26,-0.0,-0.0,1.0,(3.0343, -4.1402)},anchor=south west] (text5212) at (0.0, 0.0){$m_2$};



  \path[fill=c231f20] (3.7182, -4.5241).. controls (3.6923, -4.5286) and (3.6663, -4.5315) .. (3.6404, -4.5352).. controls (3.6341, -4.536) and (3.6274, -4.5373) .. (3.6211, -4.5376).. controls (3.6147, -4.5376) and (3.6081, -4.5371) .. (3.6018, -4.5373) -- (3.0967, -4.5373) -- (3.0967, -4.5109) -- (3.6018, -4.5109).. controls (3.6081, -4.5109) and (3.6147, -4.5109) .. (3.6211, -4.5106).. controls (3.6274, -4.5109) and (3.6341, -4.5122) .. (3.6404, -4.513).. controls (3.6663, -4.5167) and (3.6923, -4.5196) .. (3.7182, -4.5241) -- cycle;



  \path[fill=c231f20] (2.9734, -4.5241).. controls (3.0313, -4.5027) and (3.103, -4.4659) .. (3.1475, -4.4273) -- (3.1126, -4.5241) -- (3.1475, -4.6209).. controls (3.103, -4.5821) and (3.0313, -4.5455) .. (2.9734, -4.5241) -- cycle;



  \node[text=c231f20,cm={ 1.26,-0.0,-0.0,1.0,(1.9783, -4.3781)},anchor=south west] (text7757) at (0.0, 0.0){$v_{1_0}$};



  \node[text=c231f20,cm={ 1.26,-0.0,-0.0,1.0,(3.6538, -4.4994)},anchor=south west] (text2101) at (0.0, 0.0){$v_{2_0}$};



  \path[draw=c231f20,miter limit=10.0] (1.1102, -3.0689).. controls (1.0814, -3.0528) and (1.0268, -3.0279) .. (0.9583, -3.0319).. controls (0.8559, -3.0377) and (0.7461, -3.1059) .. (0.7509, -3.162).. controls (0.7551, -3.2112) and (0.8464, -3.2422) .. (0.9131, -3.265).. controls (0.9776, -3.2869) and (1.0351, -3.2935) .. (1.0766, -3.2954);



  \path[draw=c231f20,fill=c231f20,miter limit=10.0] (0.893, -3.0361).. controls (0.9136, -3.0528) and (0.9514, -3.088) .. (0.9578, -3.1366).. controls (0.9655, -3.1951) and (0.9237, -3.2385) .. (0.9046, -3.2557);



  \path[draw=c231f20,miter limit=10.0] (2.4445, -2.1796).. controls (2.4156, -2.1635) and (2.3611, -2.1386) .. (2.2926, -2.1426).. controls (2.1902, -2.1484) and (2.0804, -2.2167) .. (2.0852, -2.2728).. controls (2.0894, -2.322) and (2.1807, -2.3529) .. (2.2474, -2.3757).. controls (2.3119, -2.3977) and (2.3693, -2.4043) .. (2.4109, -2.4061);



  \path[draw=c231f20,fill=c231f20,miter limit=10.0] (2.227, -2.1468).. controls (2.2476, -2.1635) and (2.2855, -2.1987) .. (2.2918, -2.2474).. controls (2.2995, -2.3058) and (2.2577, -2.3492) .. (2.2386, -2.3664);



  \path[draw=c231f20,miter limit=10.0,dash pattern=on 0.0807cm off 0.0807cm] (4.4368, -3.9299) circle (0.4109cm);



  \path[draw=c231f20,miter limit=10.0,dash pattern=on 0.0807cm off 0.0807cm] (5.7817, -3.9672) circle (0.4109cm);



  \path[fill=c231f20] (4.0259, -3.3208).. controls (4.0391, -3.3187) and (4.0524, -3.3166) .. (4.0656, -3.3147) -- (4.0656, -3.3266).. controls (4.0524, -3.325) and (4.0391, -3.3229) .. (4.0259, -3.3205) -- cycle;



  \path[fill=c231f20] (4.1428, -3.3076) rectangle (4.2204, -3.334);



  \path[fill=c231f20] (4.2979, -3.3076) rectangle (4.3754, -3.334);



  \path[fill=c231f20] (4.4529, -3.3076) rectangle (4.5305, -3.334);



  \path[fill=c231f20] (4.608, -3.3076) rectangle (4.6477, -3.334);



  \path[fill=c231f20] (4.7704, -3.3208).. controls (4.7125, -3.3422) and (4.6408, -3.379) .. (4.5963, -3.4176) -- (4.6313, -3.3208) -- (4.5963, -3.2239).. controls (4.6408, -3.2628) and (4.7125, -3.2994) .. (4.7704, -3.3208) -- cycle;



  \node[text=c231f20,cm={ 1.26,-0.0,-0.0,1.0,(4.1534, -4.0772)},anchor=south west] (text1785) at (0.0, 0.0){$m_1$};



  \node[text=c231f20,cm={ 1.26,-0.0,-0.0,1.0,(5.5203, -4.0702)},anchor=south west] (text9526) at (0.0, 0.0){$m_2$};



  \node[text=c231f20,cm={ 1.26,-0.0,-0.0,1.0,(4.848, -3.45008)},anchor=south west] (text5620) at (0.0, 0.0){$v_{1_f}$};



  \path[fill=c231f20] (5.4091, -3.3208).. controls (5.4224, -3.3187) and (5.4356, -3.3166) .. (5.4488, -3.3147) -- (5.4488, -3.3266).. controls (5.4356, -3.325) and (5.4224, -3.3229) .. (5.4091, -3.3205) -- cycle;



  \path[fill=c231f20] (5.5264, -3.3076) rectangle (5.6039, -3.334);



  \path[fill=c231f20] (5.6814, -3.3076) rectangle (5.7589, -3.334);



  \path[fill=c231f20] (5.8364, -3.3076) rectangle (5.914, -3.334);



  \path[fill=c231f20] (5.9912, -3.3076) rectangle (6.0309, -3.334);



  \path[fill=c231f20] (6.1539, -3.3208).. controls (6.096, -3.3422) and (6.0243, -3.379) .. (5.9798, -3.4176) -- (6.0148, -3.3208) -- (5.9798, -3.2239).. controls (6.0243, -3.2628) and (6.096, -3.2994) .. (6.1539, -3.3208) -- cycle;



  \node[text=c231f20,cm={ 1.26,-0.0,-0.0,1.0,(6.2452, -3.45024)},anchor=south west] (text2694) at (0.0, 0.0){$v_{2_f}$};



\end{tikzpicture}

\end{wrapfigure}
We perform Galilean transformation.

$\bullet$ Known:
\begin{align}
    \text{System $S$ is stationary} \tag{a}\\
    \text{System $S^\prime$ is moving with velocity $u=2 \si[per-mode=symbol]{\meter \per \second}$ relative to $S$} \tag{b} \\
    m_1 = m_1^\prime,  \quad m_2 = m_2^\prime \text{~(invariance of mass)} \tag{c} \\
    \text{Motion of observer in $S^\prime$ is in the positive $x$-direction} \tag{d} \\
    \text{Direction of positive velocity} \posaxes \tag{e} \\
    v_{\text{event}}^\prime = v_{\text{event}} - u \text{~for all events (covariance of velocity)} \tag{f} \\
    m_1 v_{1_0} + m_2 v_{2_0} = m_1 v_{1_f} + m_2 v_{2_f} \text{~(conserved momentum)} \tag{g}
\end{align}
\begin{center}
    \textsl{\footnotesize Solution continued in the following page}
\end{center}
$\bullet$ Showing that the momentum is conserved under $S^\prime$ parametrically:
\begin{align}
m_1^\prime v_{1_0}^\prime + m_2^\prime v_{2_0}^\prime &= m_1^\prime v_{1_f}^\prime + m_2^\prime v_{2_f}^\prime \tag{to be proven} \\
m_1 (v_{1_0} - u) + m_2 (v_{2_0} - u)  &= m_1 (v_{1_f} - u) + m_2 (v_{2_f} - u) \text{~(Galilean transformation)} \tag{1} \\
m_1 v_{1_0} + m_2 v_{2_0} - \underline{u (m_1 + m_2)}  &= m_1 v_{1_f} + m_2 v_{2_f} - \underline{u (m_1 + m_2)}  \tag{2} \\
m_1 v_{1_0} + m_2 v_{2_0} &= m_1 v_{1_f} + m_2 v_{2_f} \tag{3} \\
\therefore \text{Momentum is conserved} \tag{from (3) and (g)}
\end{align}
$\bullet$ Alternative solution - to substitute with numbers, assume conservation of kinetic energy due to elastic collision.

In $S$:
\begin{align}
    \frac{1}{2} m_1 v^2_{1_0} + \frac{1}{2} m_2 v^2_{2_0} &= \frac{1}{2} m_1 v^2_{1_f} + \frac{1}{2} m_2 v^2_{2_f} \text{~(conserved kinetic energy)} \tag{h} \\
    0.3 \times 5 + 0.2 \times -3 &= 0.3 v_{1_f} + 0.2 v_{2_f} \text{~(substitution in (g))} \tag{i} \\
    3 v_{1_f} &= 9 - 2 v_{2_f} \tag{j}\\
    v_{1_f} &= -1.4, \quad v_{2_f} = 6.6 \text{~(from substituting with (i) in (h))} \tag{k}
\end{align}

In $S^\prime$:
\begin{align}
    m_1^\prime v_{1_0}^\prime + m_2^\prime v_{2_0}^\prime &= m_1^\prime v_{1_f}^\prime + m_2^\prime v_{2_f}^\prime \tag{to be proven} \\
    m_1 (v_{1_0} - u) + m_2 (v_{2_0} - u)  &= m_1 (v_{1_f} - u) + m_2 (v_{2_f} - u) \tag{1} \\
    0.3 \times (5-2) + 0.2 \times (-3 - 2) &= 0.3 \times (-1.4 - 2) + 0.2 \times (6.6 - 2) \tag{2} \\
    \text{Left hand side} &= -0.1 \tag{3} \\
    \text{Right hand side} &= -0.1 \tag{4} \\
    \text{Left hand side} &= \text{Right hand side} \tag{5} \\
    \therefore \text{Momentum is conserved} \tag{from (5)}
\end{align}

\begin{center}
    \rule{6cm}{0.8pt}\\
    \rule{4cm}{0.4pt}    
\end{center}
\end{document}