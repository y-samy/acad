\documentclass[12pt, answers]{exam}
\usepackage{amsmath}
\usepackage{amssymb}
\usepackage{array}
\usepackage{geometry}
\usepackage{xcolor}
\usepackage{tabularx}
\usepackage{listings}
\usepackage{adjustbox}
\usepackage[none]{hyphenat}
\usepackage{soul}

\usepackage{calc}

\geometry{a4paper, top=25mm, left=25mm, right=25mm, bottom=25mm}

% Font Settings
\usepackage{palatino}
\usepackage{anyfontsize}
\usepackage[T1]{fontenc}

\newcommand{\qtable}[1]{
    {
        \ifprintanswers
            \setlength{\linewidth}{\linewidth+4cm}
            \noindent
            \hskip-2cm
            \begin{tabularx}{\linewidth}{!{\vrule width 1.5pt}>{\hsize=1\hsize}X!{\vrule width 1.5pt}c!{\vrule width 1.5pt}>{\hsize=1\hsize}X!{\vrule width 1.5pt}c!{\vrule width 1.5pt}}
                #1
            \end{tabularx}
        \else
            \setlength{\linewidth}{\linewidth+2cm}
            \noindent
            \hskip-1cm
            \begin{tabularx}{\linewidth}{!{\vrule width 1.5pt}>{\hsize=1\hsize}X!{\vrule width 1.5pt}>{\hsize=1\hsize}X!{\vrule width 1.5pt}}
                #1
            \end{tabularx}
        \fi
    }
}

\newcommand{\choicesetter}[1]{
    & #1
}

\newcommand{\setchoice}[1]{
    \ifprintanswers
        \choicesetter{#1}
    \fi
}

\newcommand{\hsep}[1]{\noalign{\hrule height #1pt}}



\renewcommand{\questionshook}{%
    \setlength{\leftmargin}{0pt}%
    \setlength{\labelwidth}{-\labelsep}%
    \setlength{\itemsep}{0em}
    \setlength{\topsep}{0em}
}




\renewcommand{\questionlabel}{{\normalfont \thequestion.~~}}
\renewcommand{\choicelabel}{\alph{choice})}


\newcommand{\ctab}{\phantom{~~~~}}
\newcommand{\bs}{\textbackslash }
\newcommand{\n}{\newline}
\newcommand{\snippet}[1]{
    {
        \fontsize{10}{13}\selectfont
        \bfseries
        \ttfamily
        #1
    }
}
\newcommand{\mchoices}[1]{
    {
        \normalfont
        \begin{oneparchoices}
            #1
        \end{oneparchoices}
    }
}

\begin{document}



{
    \def\arraystretch{2}
    \noindent
    \begin{tabularx}{\linewidth}{!{\vrule width 1pt}>{\hsize=1.4\hsize}X!{\vrule width 1pt}>{\hsize=.6\hsize}X!{\vrule width 1pt}}
        \hsep{1}
        \textbf{Name:} Youssef Samy & \textbf{ID:} {\ifprintanswers Solutions \fi} \\
        \hsep{1}
    \end{tabularx}
}


{
\fontsize{11}{14}\selectfont
\begin{center}
    \bfseries
    \vspace{0.5cm}
    Alexandria University \hfill Faculty of Engineering\\
    \vspace{0.25cm}
    Subject Code: CSE 126 \hfill Program: CCE \\
    \underline{MODEL 1}\\
    Subject Name: Programming 1 \hfill Fall 2024 \\
    \vspace{0.25cm}
    Time Allowed: 2 Hour \hfill Final Exam (Re-created)\\
    \rule[0pt]{\linewidth}{3pt}
\end{center}
}

{
\fontsize{13}{14}\selectfont
\def\arraystretch{2}
\begin{center}
    \begin{tabular}{|W{c}{1.2cm}|W{c}{1.2cm}|W{c}{1.2cm}|W{c}{1.2cm}|}
        \hline
        Q1 & Q2 & Q3 & Total \\
        \hline
        ~~/25 & ~~/10 & ~~/15 & ~~/50\\
        \hline
    \end{tabular}
\end{center}
}

\bfseries
{
    \fontsize{12}{14}\selectfont
    \ul{Answer the Following Questions:} 3 Question, 8 Pages

    Question 1: Choose the correct answers from a, b, c and d (25 questions,
    
    25 marks).
}

\fontsize{11}{13}\selectfont
\noindent
\begin{questions}
    \qtable{
        \hsep{1.5}
        {
            \question What will be the output? \n\n
            \snippet{
                \#include <stdio.h>\n
                int main()~\{\n
                \ctab int arr[] = \{1,2,3,4,5\};\n
                \ctab int *ptr = arr + 2;\n
                \ctab printf("\%d\bs n", *ptr);\n
                \ctab return 0;
                \n\}
            }\n
            \mchoices{
                \choice 1 \qquad\qquad \choice 2\n
                \choice 3 \qquad\qquad \choice 4\n
            }
        } \setchoice{C} & 
        {
            \question A statement that reads a record from the file "trans.dat". The record consists of the integer accountNum and floating-point dollarAmount.\n\n
            \mchoices{
                \choice \snippet{nfPtr = fopen("newmast.dat", "w");}\n
                \choice \snippet{fscanf(ofPtr, "\%d\%s\%f", accountNum,\n name, \&currentBalance);}\n
                \choice \snippet{fscanf(tfPtr, "\%d\%f", \&accountNum,\n \&dollarAmount);}\n
                \choice \snippet{fprintf(nfPtr, "\%d \%s \%.2f",\n accountNum, name, currentBalance);}\n
            }
        } \setchoice{C} \\
        \hsep{1.5}
        {
            \question What will be the outputof the following code snippet? \n
            \snippet{
                char str[] = "abcdef";\n
                char *ptr = str + 2;\n
                printf("\%c\bs n", *ptr);\n
                ptr += 3;\n
                printf("\%c\bs n", *ptr);\n\n
            }
            \mchoices{
                \choice c and e \qquad\qquad\qquad \choice c and f\n
                \choice c and d \qquad\qquad\qquad \choice c and undefined\n
            }
        } \setchoice{B} &  
        {
            \question What is the printed value after the following statements\n\n
            \snippet{
                int x = 5;\n
                int y = x++ * ++x;\n
                printf("\%d", y);\n
            }\n
            \mchoices{
                \choice 25 \choice 30 \choice 35 \choice Undefined behavior \linebreak
            }
        } \setchoice{D} \\
        \hsep{1.5}
    }

\end{questions}

\end{document}