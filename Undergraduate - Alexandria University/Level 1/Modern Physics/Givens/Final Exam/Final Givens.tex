\documentclass[a4paper,12pt]{article}

% Margins
\usepackage{geometry}
\geometry{left=10mm,right=10mm,%
top=5mm,bottom=5mm}

% More Math Symbols
\usepackage{amssymb}

% Margin notes
\usepackage{marginnote}

% Dashed rules
\usepackage{dashrule}

% Trademark Symbols
\usepackage{textcomp}

% Text in math
\usepackage{amsmath}

% For Diagrams
\usepackage{graphicx}
\usepackage{wrapfig}

% Footnote URLs
\usepackage{hyperref}

% Underlines
\usepackage{soul}

% Font
\usepackage{anyfontsize}
%\usepackage[default,regular,black]{sourceserifpro}
\usepackage{tgpagella,eulervm}
\usepackage[T1]{fontenc}

% Mutli-column text
\usepackage{multicol}

% For tables
\usepackage[table]{xcolor}
\usepackage{array}
\usepackage{tablefootnote}

% Short negative symbol in math mode (sub-zero mnemonic)
\newcommand{\sz}{\text{-}}

% Clearer math
\everymath{\displaystyle}

% Degree
\usepackage{gensymb}

% Slanted Parallel
\newcommand{\slparallel}{\mathbin{\!/\mkern-5mu/\!}}

% Times ten to the power of ...
\newcommand{\tpowten}[1]{10^{#1}}
\newcommand{\tnpowten}[1]{10^{\sz#1}}
\newcommand{\ttpowten}[1]{\times 10^{#1}}
\newcommand{\ttnpowten}[1]{\times 10^{\sz#1}}


\begin{document}

% No page numbering
\pagenumbering{gobble}

% Setting the fontsize
\fontsize{12}{12}\selectfont

% Galilean Transformation
\noindent
$
    v \ll 0.1c \qquad t^\prime = t \qquad u = v_{S^\prime} - v_S \qquad x^\prime = x - ut \quad v^\prime_x = v_x - u \quad a_x^\prime = a_x \\
    y^\prime = y \qquad v^\prime_y = v_y \qquad a_y^\prime = a_y \qquad \qquad K.E = \frac{1}{2} mv^2 \qquad \rho = mv \qquad E = F d
$

{\centering \rule{18cm}{0.4pt} \par}

% Lorentz Transformation
\noindent
$
    v \ge 0.1c \qquad \gamma = \frac{1}{\sqrt{1- (u/c)^2}} > 1 \qquad t^\prime = \gamma (t - \frac{ux}{c^2}) \qquad
    x^\prime = \gamma(x - ut)\\ v^\prime_x = \frac{v_x - u}{1 - \displaystyle \frac{v_x u}{c^2} } \qquad \qquad  y^\prime = y \qquad v^\prime_y = \frac{v_y \sqrt{1-(u/c)^2}}{\displaystyle 1- \frac{v_x u}{c^2}} \qquad \theta^\prime = \tan^{\sz 1}{\frac{V_y^\prime}{V_x^\prime}} \\[-3mm]
$

{\centering \hdashrule{18cm}{0.4pt}{4pt} \par}

% Applications on Lorentz Transformation (Time dilation, length contraction, doppler shift)
\noindent
$ t_o = \gamma t_p \quad \text{rate} = \frac{1}{t} \quad L = \frac{L_p}{\gamma} \quad \theta = \tan^{\sz1}{\frac{L_y}{L_x}} \quad \rotatebox{-90}{\hskip-1cm\hdashrule{1.8cm}{0.4pt}{4pt}} \quad
    \beta = \frac{v}{c} \quad \lambda_o = \lambda_s \displaystyle \sqrt{\frac{1\pm \beta}{1\mp \beta}} \quad f_o = f_s \displaystyle \sqrt{\frac{1\mp \beta}{1\pm \beta}}\\[-.8cm]
$

{\centering \hdashrule{18cm}{0.4pt}{4pt} \par}

% Special Relativity on Mass and Energy
\noindent
$ m = m_0 \gamma \qquad E_t = \gamma m_0c^2 = m_0c^2 + K.E = \sqrt{\rho^2 c^2 + E_0^2} \qquad E_0 = m_0 c^2 \qquad K.E = (\gamma - 1) E_0 \\
    \rho = \gamma m_0v = \frac{1}{c} \sqrt{E_t^2 - E_0^2} \qquad \rho^2c^2 = E_t^2 - E_0^2 \qquad F = \gamma^3 m_0 a \qquad v = \frac{p c^2}{E_t} = c \sqrt{1- (1/\gamma^2)} $\\[-1em]

{\centering \rule{18cm}{0.4pt} \par}

% The Photo-electric Effect
\noindent
$ c = \lambda f \qquad E = hf = \frac{hc}{\lambda} = W + K.E = W + eV \qquad W = h f_c = \frac{hc}{\lambda_c} \qquad K.E_{\text{max}} = \frac{1}{2} m v^2_{\text{max}} = eV_S \qquad \\
    \frac{n}{t} = \frac{IA}{hf} = \frac{P}{E} \qquad i = \frac{n}{t} \cdot e = \frac{Q}{t} \qquad I = \frac{P}{A} \qquad E_n = n hf \qquad E_i > W \qquad f_i > f_c \qquad \lambda_i < \lambda_c
$

{\centering \rule{18cm}{0.4pt} \par}

% Blackbody Radiation
\noindent
$P = \sigma AT^4 \qquad \sigma = 5.6 \times 10^{\sz8} \qquad \lambda_{\text{max}} T = 2.898 \times 10^{-3} \text{mK} \qquad \degree \text{K} = 273 + \degree \text{C}$

{\centering \rule{18cm}{0.4pt} \par}

% Momentum - X-Ray Production - Pair Production
\noindent
$
    \lambda_{\text{min}} = \frac{hc}{eV} = \frac{1.26\times10^{\sz6}}{V} [\text{V}\cdot\text{m}] \qquad E_{x\text{-ray}} = eV = \frac{hc}{\lambda_{\text{min}}} \quad \rotatebox{-90}{\hskip-.7cm\rule{34pt}{0.4pt}} \quad E_{\text{ph}} = 2m_ec^2 + K.E_{-} + K.E_{+}\\[-4mm]
$

% Compton Scattering #TODO: add K.Emax scenarios?
{\centering \rule{18cm}{0.4pt} \par}
\noindent
$
    E = \rho c = eV \qquad \rho_\text{ph} = \frac{E_\text{ph}}{c} = \frac{h}{\lambda} \qquad \rho_e = m v \qquad K.E_e = \frac{1}{2} m v^2 = E_{\text{ph}} - E^\prime_{\text{ph}} \qquad E_\text{ph} + m_e c^2 = E_\text{ph}^\prime + E_e\\
    \lambda_c = \frac{h}{mc} = 2.426 \times 10^{\sz12} \qquad \lambda^\prime - \lambda = \lambda_c (1 - \cos{\theta}) \\
    x: \rho_i = \rho_s \cos{\theta} + \rho_e \cos{\phi} \qquad y: \rho_s \sin{\theta} = \rho_e \sin{\phi} \qquad \tan{\phi} = \frac{\sin{\theta}}{\lambda_f/\lambda_i - \cos{\theta}} = \frac{\rho^\prime \sin{\theta}}{\rho - \rho^\prime \cos{\theta}}
$

{\centering \rule{18cm}{0.4pt} \par}

% De Broglie and Uncertainty Principle
\noindent
$
    \lambda_{\text{brog}}=\frac{h}{\rho_\text{ph}}=\frac{h}{mv}=\frac{h}{\sqrt{2mK.E}} \quad hf=2\gamma mc^2 \quad \Delta x\Delta P \geq \frac{\hslash}{2} \quad \Delta E\Delta t \geq \frac{\hslash}{2} \quad \Delta \rho = m\Delta v \quad \Delta E = h\Delta f
$

{\centering \rule{18cm}{0.4pt} \par}

% Atomic Structure & Rydeberg
\noindent
$
    F_c=\frac{mv^2}{r}=F_e=\frac{1}{4\pi\epsilon_0}\frac{e}{r^2}\quad v = \frac{e}{\sqrt{4\pi\epsilon_0mr}}\quad K.E = \frac{1}{2}mv^2\quad K.E_n = \frac{e^2}{8\pi\epsilon_0r_n}\quad P.E = \frac{-e^2}{4\pi\epsilon_0r_n} \\
    E = \frac{-e^2}{8\pi\epsilon_0r_n} \quad r_n = a_0n^2=\frac{n^2h^2\epsilon_0}{\pi me^2} \qquad L = n \hslash = \rho r \qquad f_n = \frac{v}{2\pi r} = \frac{e}{2\pi\sqrt{4\pi\epsilon_0mr^3}} = \frac{-E_1}{h}(\frac{2}{n^3})\\[+2mm]
    E_n = \frac{-13.6}{n^2} [e\text{V}] \quad \frac{1}{\lambda} = R_\infty(\frac{1}{n_f^2}-\frac{1}{n_i^2}) \quad E_\text{ph} = hf = E_i - E_f \quad N = f\Delta t \quad f = \frac{-E_1}{h}(\frac{1}{n_f^2}-\frac{1}{n_i^2}) \quad f_L = \frac{-E_1}{h}(\frac{2\rho}{n^3})
$

{\centering \rule{18cm}{0.4pt} \par}

% Quantization, Zeeman, Energy Levels, etc...
\noindent
$
    K.E = \frac{\rho^2}{2m} \quad  \hslash = \frac{h}{2\pi}\quad  \mu = -(\frac{e}{2m})L\quad P.E_m = \sz \mu B\cos{\theta}\quad  \Delta\lambda = \frac{eB\lambda^2}{4\pi mc} \quad \Delta v = \frac{eB}{4\pi m}\\[+1mm]
    L_z = m_l\hslash = L\cos{\theta} \quad\cos{\theta} = \frac{m_l}{\sqrt{l(l+1)}}\quad S_z = \pm\frac{1}{2}\hslash \quad \mu_s = \frac{-e}{m}S \quad \mu_{sz} = \pm\frac{e\hslash}{2m}=\pm\mu_B l = n-1 \\
    n = 1,2,3,4\dots\quad l = 0,1,\dots,n-1\quad -l\leq m_l\leq l\quad m_s = \pm\frac{1}{2} \quad  L = \sqrt{l(l+1)}\hslash \quad S = \frac{\sqrt{3}}{2} \hslash \quad J = L+S\\
    N_{\text{max}} = 2n^2\quad L_{\text{max}} = 2(2l+1)\qquad \Delta l=\pm1\quad \Delta m_l=0,\pm1 \\
    1s^2 \quad 2s^2 \quad 2p^6 \quad 3s^2 \quad 3p^6 \quad 4s^2 \quad 3d^{10} \quad 4p^6 \quad 5s^2 \quad 4d^{10} \quad 5p^6 \quad 6s^2 \quad 4f^{14} \quad 5d^{10} \quad 6p^6 \quad 7s^2 \quad 6d^{10} \quad 5f^{14}
$

{\centering \rule{18cm}{0.4pt} \par}


% Constants and Conversions
\noindent
$
    c = 3 \times 10^8 \text{ m}/\text{s} \qquad \text{Mach} = 343 \text{ m}/\text{s} \qquad 1.6\times 10^{\sz 19}\text{ J} = 1 \ e\text{V} \qquad e = 1.6 \ttnpowten{19} \text{ C}\\[+1mm]
    m_{e} = 9.11 \times 10^{\sz31} \text{ kg} \quad m_{\text{p}} = 1.67 \times 10^{\sz27} \text{ kg} \qquad \text{M}e\text{V}/c^2 = 1.79 \times 10^{\sz30} \text{ kg} \qquad \text{M}e\text{V}/c = 5.36 \times 10^{\sz22} \text{ kg}\cdot\text{m}/\text{s} \\
    m_e c^2 = 0.511 \text{ M}e\text{V} \qquad m_\text{p}c^2 = 938 \text{ M}e\text{V} \qquad m_\text{n}c^2 = 939 \text{ M}e\text{V}\\
    \mu_B = 9.274 \ttnpowten{24} \text{J}/\text{T} = 5.788 \ttnpowten{5} e\text{V}/ \text{T} \qquad \frac{h}{m_e c} = 0.024 \ttnpowten{10} \text{ m}\\
    h = 6.626 \ttnpowten{34} \text{ J} \cdot \text{s} \qquad \hslash = 1.054 \ttnpowten{34} \text{ J} \cdot \text{s} \qquad a_0 = 0.529 \ttnpowten{10} \text{ m} \qquad R = 1.097\ttpowten{7} \text{ m}^{\sz1}\\
    1 \text{ L.Y} \approx 9.46 \ttpowten{15} \text{ m} \qquad \mu\text{m}: \tnpowten{6} \qquad \text{nm} : \tnpowten{9} \qquad \text{pm} : \tnpowten{12} \qquad \text{fm} : \tnpowten{15} \qquad \text{\AA} = 10^{\sz10} \text{ m}\\
$
\end{document}